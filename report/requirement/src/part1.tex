\newpage
\section{引言}
\subsection{编写目的}
撰写本篇需求文档的目的如下:
\begin{enumerate}
	\item 明确项目需求,对该系统进行一个能够达成一致认可的描述,明确需要完成的功能;
	\item 明确开发资源与项目目标;
	\item 控制整个开发过程,作为开发手册的一部分,降低开发中小组讨论的成本。
\end{enumerate}

文档预期读者包括开发人员、任务提出者及其他使用该资源的用户。
\subsection{背景}
本项目的系统名称为32位MIPS处理器。

本项目任务由计算机组成原理课程刘卫东老师、李山山老师和软件工程课程白晓颖老师共同提出。

承担本项目的开发者为计33班的徐炜杰、王楠和黄欢。此外还受到了刘卫东、白晓颖、李山山三位老师的指导和王钧奕、张乐两位助教的帮助。
\subsection{定义}
本段中给出了需求文档中对于一些术语的基本定义,具体如下。

\begin{table}[H]
	\centering
	\caption{定义列表}
	\begin{tabular}{|c|l|}
		\hline
		术语 & 描述 \\
		\hline
		RISC & Reduced Instruction Set Computer的缩写,指精简CPU指令集 \\
		\hline
		\multirow{2}{*}{MIPS} & Microprocessor without Interlocked piped stages的缩写,指无内部互锁流水级的微 \\
		& 处理器,为一种典型RISC指令集 \\
		\hline
		CPU & 中央处理器,为本项目的最终目标 \\
		\hline
		\multirow{2}{*}{CP0} & 系统控制协处理器,是CPU和操作系统交互的窗口;CPU通过CP0向操作系统传递运 \\
		& 行信息和异常信息等,而操作系统则通过操作CP0设置硬件的运行状态 \\
		\hline
		\multirow{2}{*}{MMU} & 内存管理单元,负责对虚拟地址进行映射,并总管访问各种存储器和输入输出接口的 \\
		& 功能 \\
		\hline
		\multirow{2}{*}{TLB} & 传输后备缓冲器(快表),即页表的高速缓存,用于加快地址转换。本项目需对内存 \\
		& 进行分页管理,即虚拟地址通过页表映射得到物理地址。 \\
		\hline
		ALU & 算术逻辑单元 \\
		\hline
		RAM & 存储程序的硬件,断电不保存信息,速度较快,常用于电脑主存储器 \\
		\hline
		ROM & 存储程序的硬件,断电可保存信息 \\
		\hline
		Flash & 存储程序的硬件,断电可保存信息,实验中用作硬盘 \\
		\hline
		BIOS & 主板上的启动程序,负责初始化硬件引导操作系统 \\
		\hline
		\multirow{2}{*}{BootLoader} & 由BIOS启动的一段特殊程序,用于将操作系统从Flash中加载到内存中,并且跳转到 \\
		& 开始地址执行 \\
		\hline
		\multirow{2}{*}{Controller} & CPU模块之一,根据指令和硬件反馈信息给出控制信号,控制各个部件和整个CPU \\
		& 的运行 \\
		\hline
		组合逻辑元件 & 与运行时钟无关的元件,某一时刻的输出只取决于当前输入 \\
		\hline
		\multirow{3}{*}{时序逻辑元件} & 根据系统的时钟信号,在时钟的上升沿工作的元件。某一时刻的稳定输出由当前输入 \\
		& 和历史状态共同决定。注意组合逻辑元件和时序逻辑元件往往没有明确的分界;一个 \\
		& 寄存器往往仅在上升沿写入,但却随时按照组合逻辑方式将其内容进行输出\\
		\hline
	\end{tabular}
\end{table}

\subsection{参考资料}

\begin{enumerate}

\item 实验指导文档

\item OSLab实验参考文档
	
\item 贾开, 周昕宇, 李铁铮, 等. 计算机组成原理综合实验报告

\item 逆光组. 清华大学计算机组成原理32位Mips CPU教程

\item 刘卫东, 李山山, 宋佳兴, 等. 计算机硬件系统实验教程[M]. 清华大学出版社, 2013.
	
\item 帕特森, 亨尼斯, 康继昌, 等. 计算机组成与设计: 硬件/软件接口[M]. 机械工业出版社, 2012.
	
\item Sweetman D. See MIPS run[M]. Morgan Kaufmann, 2010.

\item 雷思磊. 自己动手写CPU[M]. 电子工业出版社, 2014.

\end{enumerate}
