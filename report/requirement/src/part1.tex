\newpage
\section{引言}
\subsection{编写目的}
计算机组成原理32位MIPS实验是在计原16位实验的基础上的扩展。在实验原理方面与软件工程、操作系统、编译原理等多门课程相结合。实验初期学习曲线较陡,在原理方面比较难以掌控。在编程实现方面涉及到大规模VHDL代码的书写,需要对数字逻辑设计有比较清晰的思路,工作量非常大。

撰写这篇需求文档的目的在于以下几点:
\begin{enumerate}
	\item 明确项目需求,对该系统进行一个能够达成一致认可的描述,明确需要完成的功能;
	\item 明确开发资源与项目目标;
	\item 控制整个开发过程,作为开发手册的一部分,降低开发中小组讨论的成本。
\end{enumerate}

文档预期读者包括开发人员、任务提出者及其他使用该资源的用户。
\subsection{背景}
本项目的系统名称为32位MIPS处理器。

本项目任务由计算机组成原理课程刘卫东老师、李山山老师和软件工程课程白晓颖老师共同提出。

承担本项目的开发者为计33班的徐炜杰、王楠和黄欢。此外还受到了刘卫东、白晓颖、李山山三位老师的指导和王钧奕、张乐两位助教的帮助。
\subsection{定义}
本段中给出了需求文档中对于一些术语的基本定义,具体如下。

\begin{table}[H]
	\centering
	\caption{定义列表}
	\begin{tabular}{|c|l|}
		\hline
		术语 & 描述 \\
		\hline
		RISC & Reduced Instruction Set Computer的缩写,指精简CPU指令集 \\
		\hline
		\multirow{2}{*}{MIPS} & Microprocessor without Interlocked piped stages的缩写,指无内部互锁流水级的微 \\
		& 处理器,为一种典型RISC指令集 \\
		\hline
		CPU & 中央处理器,为本项目的最终目标 \\
		\hline
		\multirow{2}{*}{CP0} & 系统控制协处理器,是CPU和操作系统交互的窗口;CPU通过CP0向操作系统传递运 \\
		& 行信息和异常信息等,而操作系统则通过操作CP0设置硬件的运行状态 \\
		\hline
		\multirow{2}{*}{MMU} & 内存管理单元,负责对虚拟地址进行映射,并总管访问各种存储器和输入输出接口的 \\
		& 功能 \\
		\hline
		\multirow{2}{*}{TLB} & 传输后备缓冲器(快表),即页表的高速缓存,用于加快地址转换。本项目需对内存 \\
		& 进行分页管理,即虚拟地址通过页表映射得到物理地址。 \\
		\hline
		ALU & 算术逻辑单元 \\
		\hline
		RAM & 存储程序的硬件,断电不保存信息,速度较快,常用于电脑主存储器 \\
		\hline
		ROM & 存储程序的硬件,断电可保存信息 \\
		\hline
		Flash & 存储程序的硬件,断电可保存信息,实验中用作硬盘 \\
		\hline
		BIOS & 主板上的启动程序,负责初始化硬件引导操作系统 \\
		\hline
		\multirow{2}{*}{BootLoader} & 由BIOS启动的一段特殊程序,用于将操作系统从Flash中加载到 \\
		& 内存中,并且跳转到开始地址执行 \\
		\hline
		\multirow{2}{*}{Controller} & CPU模块之一,根据指令和硬件反馈信息给出控制信号,控制各个部件和整个CPU \\
		& 的运行 \\
		\hline
		组合逻辑元件 & 与运行时钟无关的元件,某一时刻的输出只取决于当前输入 \\
		\hline
		\multirow{3}{*}{时序逻辑元件} & 根据系统的时钟信号,在时钟的上升沿工作的元件。某一时刻的稳定输出由当前输入 \\
		& 和历史状态共同决定。注意组合逻辑元件和时序逻辑元件往往没有明确的分界;一个 \\
		& 寄存器往往仅在上升沿写入,但却随时按照组合逻辑方式将其内容进行输出\\
		\hline
	\end{tabular}
\end{table}
\subsection{Mips架构下CPU运行概述}
本段给出MIPS架构下CPU运行的基本方式。这是设计MIPS架构CPU需要掌握的基础知识,也是以下需求分析的基础,望读者详加掌握。

MIPS架构下的计算机和所有计算机一样,基本功能都是从内存中提取指令并执行,保存所提取的指令地址的寄存器称为PC。围绕这一功能,定义了合适的体系结构实现方案。首先,硬件运行分为用户态和内核态以进行权限管理。二者的不同之处是,内核态可使用的内存空间更大,可使用的指令条数和硬件模块更多,地址映射方法也和用户态不同。之所以如此设计,是为了方便操作系统完成TLB管理、异常处理等系统级操作,并将以上内容向用户隐藏。如此既可方便用户,也防止了用户的误操作造成危险。而在此过程中需要用到的,就是协处理器CP0。如在异常处理过程中,硬件检测到异常后产生详细信息,将其与发生异常的指令地址一起保存在CP0寄存器中,并跳转到操作系统的通用异常处理入口处。随后该段代码通过检查CP0的保存数据识别到发生的异常,并分发到对其进行处理的代码段。完成处理后,CPU跳转回被打断的指令继续执行,如此即完成了一次异常处理。内存管理方面采用虚拟地址机制,将虚拟地址映射为物理地址、对应到实际的存储模块。如此即可方便地设置权限,防止访问未经授权的资源而造成险情。同时使得各个进程之间互不干扰,虚拟地址可以复用,大大简化了程序的设计。MIPS中从虚拟地址到物理地址的映射方法有多种,而方法的选择则是依据虚拟地址的值本身。其中用户态下虚拟地址的映射较为复杂,需要通过查表进行对应,这就是所谓的页表管理机制。考虑到该查找操作非常频繁,而页表本身存放在内存中,对其进行访问代价很大,所以选择将页表的一部分放入CPU做成高速缓存,这就是快表TLB。最后,访存操作还为CPU和外界的互动提供了统一接口,如从串口的输入输出就是通过读写某一特定地址来实现的。硬件上总管从接收虚拟地址开始映射到读写数据完成这整个过程的模块就是MMU。

\subsection{参考资料}
[1]实验指导文档

[2]OSLab实验参考文档
	
[3]计算机组成原理综合实验报告 贾开

[4]刘卫东, 李山山, 宋佳兴, 等. 计算机硬件系统实验教程[M]. 清华大学出版社, 2013.
	
[5]帕特森, 亨尼斯, 康继昌, 等. 计算机组成与设计: 硬件/软件接口[M]. 机械工业出版社, 2012.
	
[6]Sweetman D. See MIPS run[M]. Morgan Kaufmann, 2010.

[7]雷思磊. 自己动手写CPU[M]. 电子工业出版社, 2014.
