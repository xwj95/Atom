\section{功能需求}
\subsection{CPU}
\subsubsection{ALU}
\subsubsection{乘法器}
\subsubsection{寄存器堆}
\subsubsection{CP0}
\subsubsection{异常中断处理}
\subsubsection{MMU}
\paragraph{虚拟地址映射}
\paragraph{TLB}

\subsection{Ucore}
\subsubsection{BIOS}
BIOS即为启动ucore所用的Bootloader程序,通常是放在Flash中。而本实验中我们将在FPGA里建立一块ROM,将Bootloader放置在该ROM中,并且设置CPU的访问地址从该ROM开始。这样能避免由于Flash的读写不稳定而对BIOS造成的破坏,还能将ucore与其独立开来。

BIOS启动时,Flash中的操作系统加载到内存中,然后跳转到操作系统的初始化代码,从而开始操作系统的工作。
\subsubsection{远程文件执行}
实验要求修改ucore,实现简单的远程文件执行功能,即通过串口从PC上获取ELF文件,并在本地执行。
\subsection{外设}
\subsubsection{串口}
串口模块的主要部分位于板子上的CPLD中,在FPGA端,对串口的控制通过data$\_$ready、tbre、\\
tsre、rdn、wrn进行。

当data$\_$ready=‘1’时串口数据就绪,可以读出。

当tbre and tsre=‘1’是,表示可以向串口写入。

将rdn置‘0’且数据线写高阻,可以从数据线得出串口数据。

将数据写入数据线且将wrn置‘0’,可以向串口发出数据。

串口模块1FD003F8地址表示数据(只有低8位有效),必须使用SW指令写入。

1FD003FC地址表示状态寄存器Status,Status$\&$1=1时可写,Status$\&$2=2时可读。
\subsubsection{VGA}
Device$\_$VGA模块是显示控制模块,接受CPU写入的ASCII码数据,并维护一个字符矩阵,以字符中断的形式通过VGA接口输出到显示器。

CPU需要从1FC03000地址读出数据,如果结果为‘1’表示可以写入,否则不能吸入;可以写入时,向1FC03000地址写入ASCII数据即可。
\subsubsection{ps/2键盘}
Device$\_$Keyboard模块位键盘控制模块。只要从0F000000地址读出数据,结果为0表示没有新数据,否则读到的就是键入的ASCII码。

PS/2键盘读到的是扫描码,为了方便软件,可以在硬件层面加入编码转换,CPU读到的直接就是ASCII码。
\subsubsection{网口}
实验提供的网口是DM9000A,它带有通用处理器接口的以太网控制器,一个10/100M PHY和4K双字的SRAM,IO端口支持3.3V与5V容限值。

在FPGA中需要做地址映射。
\subsection{Decaf编译器}
\subsubsection{汇编指令的生成}
由于简化的CPU中并未实现add、sub指令,需要把decaf的MIPS后端里生成add、sub指令的部分改成addu、subu,区别仅在于溢出时后者不会产生异常。

另外、CPU中也未实现除法指令,不过由于所用测试程序中没有除法运算,因此不进行相关修改。如果需要除法,可以用其它指令手动实现除法函数,并把除法翻译成函数调用。
\subsubsection{库函数调用及calling convention}
标准MIPS32使用032 ABI,函数调用的前四个参数通过\$a0-\$a3四个寄存器传输;但decaf编译出的程序的参数全都在栈上传递。当然,无论什么calling convention,只要能自恰,程序本身就应该能正常运行,所以需要解决的问题只有用于程序与C实现的库函数及操作系统交互的部分。

在这里,我们在decaf和库函数之间增加一个适配器层,将decaf的调用约定翻译成032 ABI再调用库函数。
\subsubsection{程序入口及退出}
直接使用ucore里的linker script(user.Id)以及用户静态函数库libuser.a,在该环境下系统会设置好一些全局变量,然后跳转到main执行。我们将修改decaf编译器,将其输出的main重命名为decaf$\_$main,然后汇编实现一个新的main函数。由于decaf的main是void类型,我们默认其执行成功返回0,于是在decaf$\_$main返回后直接调用exit(0)。
\subsection{指令集与数据通路}
