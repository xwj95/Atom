\section{功能需求}
\subsection{CPU}
\subsubsection{ALU}
ALU负责实现双输入的算术、逻辑和移位运算功能,其中包括乘法运算,并完成指令系统的算术指令。

ALU的输入端为两个32位整数和一个8位操作码,输出端为32位整数(乘法运算除外),不给出标志位。

由于ALU需要实现的运算种类较多,我们在此不做列举。

%\begin{table}[ht]
%\centering
%\begin{tabular}{cccccc}
%\hline
%操作码&功能&描述&操作码&功能&描述\\
%\hline
%ADD&A + B&加法&NOR&$\sim$(A $\mid$ B)&或非\\
%SUB&A - B&减法&SLL&A $\gg$ B&逻辑左移\\
%AND&A $\&$ B&与&SRL&A $\ll$ B&逻辑右移\\
%OR&A $\mid$ B&或&SRA&A $\ggg$ B&算术右移\\
%XOR&A $\wedge$ B&异或&SLT&A $<$ B&比较\\
%\hline
%\end{tabular}
%\end{table}

\subsubsection{乘法器}
乘法器负责实现乘法功能,输入为两个32位整数,输出为两个32位整数,分别存放在LO和HI寄存器中。

乘法运算采用Verilog语言提供的乘法运算符实现。我们没有将乘法器作为一个独立的元件来实现,而是在ALU中加入乘法的操作码,将乘法器并入ALU中实现。

考虑到乘法运算需要的时间比较长,如果乘法指令需要多周期完成,那么就必须为流水线增加暂停机制,这将使流水线的设计复杂化。为尽可能简化流水线设计,可适当降低时钟频率,以使得乘法运算可以在一个时钟周期内完成运算。

\subsubsection{寄存器堆}
寄存器堆负责实现通用寄存器的读写和在数据通路中的控制,在流水线译码阶段读取一个或两个通用寄存器的数据(组合逻辑),并在流水线写回阶段将结果写入通用寄存器(时序逻辑)。

寄存器堆采用FPGA的逻辑单元来实现数据的存储,在32位MIPS架构下需要实现32个32位通用寄存器。

\subsubsection{CP0}
CP0是系统控制协处理器,本次实验我们需要通过CP0实现对TLB、MMU及异常处理的管理机制。

下表列出了CP0的寄存器及其功能,这里我们只给出必须实现的11个寄存器。

\begin{table}[H]
\centering
\begin{tabular}{lll}
\hline
编号&寄存器名称&寄存器功能\\
\hline
0&Index&用于TLBWI指令访问TLB入口的索引序号\\
2&EntryLo0&作为TLBWI及其他TLB指令接口,管理偶数页入口\\
3&EntryLo1&作为TLBWI及其他TLB指令接口,管理奇数页入口\\
9&BadVAddr&捕捉最近一次地址错误或TLB异常(重填、失效、修改)时的虚拟地址\\
10&Count&每隔一个时钟增加1,用作计数器,并可使能控制\\
11&EntryHi&TLB异常时,系统将虚拟地址部分写入EntryHi寄存器中用于TLB匹配信息\\
12&Compare&保持一定值,当Count值与Compare相等时,SI$\_$TimerInt引脚变高电平直到有数值写入Compare,用于定时中断\\
13&Status&表示处理器的操作模式、中断使能及诊断状态\\
15&Cause&记录最近一次异常的原因,控制软件中断请求以及中断处理派分的向量\\
16&EPC&存储异常处理之后程序恢复执行的地址\\
18&EBase&识别多处理器系统中不同的处理器异常向量的基地址\\
\hline
\end{tabular}
\end{table}

\begin{enumerate}[(1)]
\item Index寄存器
Index寄存器是一个32位读/写寄存器,可用于TLBP、TLBR和TLBWI指令访问TLB入口的索引序号。

Index区域的大小根据具体实现方式随TLB的入口个数而定。对于基于TLB的内存管理单元MMU来说,该区域的最小值为$\lceil log_{2}(TLBEntries)\rceil$。

如果一个写入Index寄存器的值大于等于TLB入口数,则该处理器的操作是未定义的。

该寄存器仅对TLB有效。

\item EntryLo1/EntryLo0寄存器
这对EntryLo寄存器的作用等同于TLB、TLBR、TLBWI和TLBWR指令间的接口。对基于TLB的MMU而言,EntryLo0管理偶数页的入口,EntryLo1管理奇数页的入口。如果出现了地址错误,TLB失效,TLB修改或是TLB重填异常的行为,那么EntryLo0和EntryLo1寄存器的内容将会成为未定义的。只有当基于TLB的存储管理单元存在时,这些寄存器才有效。
\item EntryHi寄存器
\item Status寄存器
\item Cause寄存器
\item Ebase寄存器
\end{enumerate}

\subsubsection{异常中断处理}
\subsubsection{MMU}
\paragraph{虚拟地址映射}
\paragraph{TLB}

\subsection{Ucore}
\subsubsection{BIOS}
BIOS即为启动ucore所用的Bootloader程序,通常是放在Flash中。而本实验中我们将在FPGA里建立一块ROM,将Bootloader放置在该ROM中,并且设置CPU的访问地址从该ROM开始。这样能避免由于Flash的读写不稳定而对BIOS造成的破坏,还能将ucore与其独立开来。

BIOS启动时,Flash中的操作系统加载到内存中,然后跳转到操作系统的初始化代码,从而开始操作系统的工作。
\subsubsection{远程文件执行}
实验要求修改ucore,实现简单的远程文件执行功能,即通过串口从PC上获取ELF文件,并在本地执行。
\subsection{外设}
\subsubsection{串口}
串口的功能需求为实现与PC机的通信,通过计算机键盘输入数据,向计算机输出数据。

串口模块的主要部分位于板子上的CPLD中,在FPGA端,对串口的控制通过data$\_$ready、tbre、\\
tsre、rdn、wrn进行。

当data$\_$ready=‘1’时串口数据就绪,可以读出。

当tbre and tsre=‘1’是,表示可以向串口写入。

将rdn置‘0’且数据线写高阻,可以从数据线得出串口数据。

将数据写入数据线且将wrn置‘0’,可以向串口发出数据。

串口模块1FD003F8地址表示数据(只有低8位有效),必须使用SW指令写入。

1FD003FC地址表示状态寄存器Status,Status$\&$1=1时可写,Status$\&$2=2时可读。
\subsubsection{VGA}
Device$\_$VGA模块是显示控制模块,接受CPU写入的ASCII码数据,并维护一个字符矩阵,以字符中断的形式通过VGA接口输出到显示器。

CPU需要从1FC03000地址读出数据,如果结果为‘1’表示可以写入,否则不能吸入;可以写入时,向1FC03000地址写入ASCII数据即可。
\subsubsection{ps/2键盘}
Device$\_$Keyboard模块位键盘控制模块。只要从0F000000地址读出数据,结果为0表示没有新数据,否则读到的就是键入的ASCII码。

PS/2键盘读到的是扫描码,为了方便软件,可以在硬件层面加入编码转换,CPU读到的直接就是ASCII码。
\subsubsection{网口}
这是拓展要求。实验提供了DM9000A网口芯片与PC机进行网络通讯。该芯片带有通用处理器接口的以太网控制器,一个10/100M PHY和4K双字的SRAM,IO端口支持3.3V与5V容限值。

当收到数据包时,芯片通过中断信号方式通知CPU触发异常,由操作系统对数据包进行处理。若要完成此需求,则需对操作系统进行改写:在初始化时添加网口中断使能、添加网口芯片初始化代码、手动实现网络通信协议和网口驱动等。
\subsection{Decaf编译器}
\subsubsection{汇编指令的生成}
由于简化的CPU中并未实现add、sub指令,需要把decaf的MIPS后端里生成add、sub指令的部分改成addu、subu,区别仅在于溢出时后者不会产生异常。

另外、CPU中也未实现除法指令,不过由于所用测试程序中没有除法运算,因此不进行相关修改。如果需要除法,可以用其它指令手动实现除法函数,并把除法翻译成函数调用。
\subsubsection{库函数调用及calling convention}
标准MIPS32使用032 ABI,函数调用的前四个参数通过\$a0-\$a3四个寄存器传输;但decaf编译出的程序的参数全都在栈上传递。当然,无论什么calling convention,只要能自恰,程序本身就应该能正常运行,所以需要解决的问题只有用于程序与C实现的库函数及操作系统交互的部分。

在这里,我们在decaf和库函数之间增加一个适配器层,将decaf的调用约定翻译成032 ABI再调用库函数。
\subsubsection{程序入口及退出}
直接使用ucore里的linker script(user.Id)以及用户静态函数库libuser.a,在该环境下系统会设置好一些全局变量,然后跳转到main执行。我们将修改decaf编译器,将其输出的main重命名为decaf$\_$main,然后汇编实现一个新的main函数。由于decaf的main是void类型,我们默认其执行成功返回0,于是在decaf$\_$main返回后直接调用exit(0)。
\subsection{指令集与数据通路}
