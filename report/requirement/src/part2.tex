\section{功能需求}
\subsection{CPU}
\subsubsection{ALU}
ALU负责实现双输入的算术、逻辑和移位运算功能,其中包括乘法运算,并完成指令系统的算术指令。

ALU的输入端为两个32位整数和一个8位操作码,输出端为32位整数(乘法运算除外),不给出标志位。

由于ALU需要实现的运算种类较多,我们在此不做列举。

%\begin{table}[ht]
%\centering
%\begin{tabular}{cccccc}
%\hline
%操作码&功能&描述&操作码&功能&描述\\
%\hline
%ADD&A + B&加法&NOR&$\sim$(A $\mid$ B)&或非\\
%SUB&A - B&减法&SLL&A $\gg$ B&逻辑左移\\
%AND&A $\&$ B&与&SRL&A $\ll$ B&逻辑右移\\
%OR&A $\mid$ B&或&SRA&A $\ggg$ B&算术右移\\
%XOR&A $\wedge$ B&异或&SLT&A $<$ B&比较\\
%\hline
%\end{tabular}
%\end{table}

\subsubsection{乘法器}
乘法器负责实现乘法功能,输入为两个32位整数,输出为两个32位整数,分别存放在LO和HI寄存器中。

乘法运算采用Verilog语言提供的乘法运算符实现。我们没有将乘法器作为一个独立的元件来实现,而是在ALU中加入乘法的操作码,将乘法器并入ALU中实现。

考虑到乘法运算需要的时间比较长,如果乘法指令需要多周期完成,那么就必须为流水线增加暂停机制,这将使流水线的设计复杂化。为尽可能简化流水线设计,可适当降低时钟频率,以使得乘法运算可以在一个时钟周期内完成运算。

\subsubsection{寄存器堆}
寄存器堆负责实现通用寄存器的读写和在数据通路中的控制,在流水线译码阶段读取一个或两个通用寄存器的数据(组合逻辑),并在流水线写回阶段将结果写入通用寄存器(时序逻辑)。

寄存器堆采用FPGA的逻辑单元来实现数据的存储,在32位MIPS架构下需要实现32个32位通用寄存器。

\subsubsection{CP0}
CP0是系统控制协处理器,本次实验我们需要通过CP0实现对TLB、MMU及异常处理的管理机制。

下表列出了CP0的寄存器及其功能,这里我们只给出必须实现的11个寄存器。

\begin{table}[H]
\centering
\begin{tabular}{lll}
\hline
编号&寄存器名称&寄存器功能\\
\hline
0&Index&用于TLBWI指令访问TLB入口的索引序号\\
2&EntryLo0&作为TLBWI及其他TLB指令接口,管理偶数页入口\\
3&EntryLo1&作为TLBWI及其他TLB指令接口,管理奇数页入口\\
9&BadVAddr&捕捉最近一次地址错误或TLB异常(重填、失效、修改)时的虚拟地址\\
10&Count&每隔一个时钟增加1,用作计数器,并可使能控制\\
11&EntryHi&TLB异常时,系统将虚拟地址部分写入EntryHi寄存器中用于TLB匹配信息\\
12&Compare&保持一定值,当Count值与Compare相等时,SI$\_$TimerInt引脚变高电平直到有数值写入Compare,用于定时中断\\
13&Status&表示处理器的操作模式、中断使能及诊断状态\\
15&Cause&记录最近一次异常的原因,控制软件中断请求以及中断处理派分的向量\\
16&EPC&存储异常处理之后程序恢复执行的地址\\
18&EBase&识别多处理器系统中不同的处理器异常向量的基地址\\
\hline
\end{tabular}
\end{table}

\begin{enumerate}[(1)]
\item Index寄存器
Index寄存器是一个32位读/写寄存器,可用于TLBP、TLBR和TLBWI指令访问TLB入口的索引序号。

Index区域的大小根据具体实现方式随TLB的入口个数而定。对于基于TLB的内存管理单元MMU来说,该区域的最小值为$\lceil log_{2}(TLBEntries)\rceil$。

如果一个写入Index寄存器的值大于等于TLB入口数,则该处理器的操作是未定义的。

该寄存器仅对TLB有效。

\item EntryLo1/EntryLo0寄存器
这对EntryLo寄存器的作用等同于TLB、TLBR、TLBWI和TLBWR指令间的接口。对基于TLB的MMU而言,EntryLo0管理偶数页的入口,EntryLo1管理奇数页的入口。如果出现了地址错误,TLB失效,TLB修改或是TLB重填异常的行为,那么EntryLo0和EntryLo1寄存器的内容将会成为未定义的。只有当基于TLB的存储管理单元存在时,这些寄存器才有效。

\item EntryHi寄存器
EntryHi寄存器包含了用于TLB读、写和访问操作的虚拟地址匹配信息。当TLB异常(TLB Refill,TLB Invalid或TLB Modified)发生时,系统将虚拟地址的[31:13]位写入EntryHi寄存器的VPN2区域。TLBR指令将选中的TLB入口相应的区域写入EntryHi寄存器。软件(通常是操作系统)将当前地址空间标识符写入ASID区域,该区域在TLB比较过程中用于确定TLB是否可以匹配。

由于ASID区域被TLBR指令重填覆盖了,软件必须保存和重新存储有关TLBR使用的ASID的值。这在发生TLB失效和TLB修改异常时,以及在其它存储管理软件中尤为重要。

在发生了地址错误的异常后,EntryHi寄存器的VPN2区域将成为未定义的,并且该区域可能在发生地址错误异常的过程中被硬件修改。EntryHi寄存器的软件写操作(通过MTC0)不会导致BadVAddr和Context寄存器中的地址相关区域发生隐式的写入(implicit write)。

该寄存器仅对TLB有效。

\item Status寄存器
Status寄存器是一个读/写寄存器,可以表示处理器的操作模式、中断使能以及诊断状态。该寄存器的区域联合作用,可以创建处理器的工作模式。

中断使能:当一下所有条件成立时启用中断:

\quad Status[0]:IE = 1

\quad Status[0]:EXL = 0

\quad Status[0]:ERL = 0

\quad 额外的:Debug[0]:DM = 0

当这些条件都符合时,设置IM(Status[16:9])位和IE位可以使能中断。

EXL与ERL任一位置1都可使系统进入Kernel模式,否则为User模式。

\item Cause寄存器
Cause寄存器主要记录最近一次异常的原因,也控制软件中断请求以及中断处理派分的向量。除了IP1..0、DC、IV和WP区域,Cause寄存器中其它的所有区域都是只读的。第二版架构在外部中断控制器(EIC)的中断模式下,增加了可选项。在这个模式下,IP7..2表示请求中断优先级(RIPL)。

Cause[6:2]表示ExcCode,即异常号。

\item Ebase寄存器
EBase寄存器是一个读/写寄存器,包含了在StatusBEV为0时所使用的异常向量的基地址及一个只读CPU号,该CPU号可以被软件用来区分多处理器系统中不同的处理器。

EBase寄存器使软件能在多处理器系统中识别特定的处理器,并允许每个处理器的异常向量可以不同,特别是对由多种处理器组成的系统。当StatusBEV为0时,EBase寄存器的位31:12由0构成,形成异常向量的基地址。当StatusBEV为1时,或在任何EJTAG调试异常的情况下,异常向量的基地址由固定的缺省值确定。

EBase寄存器的位31:30固定为2\#10,从而强制异常基地址在kseg0或kseg1的无映射的虚拟地址段中。在缓存错误异常中,异常基地址的位29将被强制置为1,从而使异常处理器从无缓存的kseg1段开始执行。如果需要改变异常基地址寄存器的值,则操作必须在StatusBEV为1的情况下进行。如果在StatusBEV为0时,在异常基地址区域写入了一个不同的值,处理器的操作将成为未定义的。

将位31:12与异常基地址区域相结合,可允许将异常向量的基地址置于任何4KB字节大小的页面的边界上。

\end{enumerate}

通过调用MFC0,MTC0指令,CP0提供了统一的对外接口以完成对寄存器组的访问。\\\\
实现步骤:

\begin{enumerate}[(1)]
\item 选择要实现的CP0寄存器,可考虑推荐实现的寄存器或自行决定额外寄存器实现。

\item 按照CP0寄存器的功能分别在CPU的不同模块完成各寄存器的赋值。

\begin{enumerate}[(a)]
\item 在译码、运算及访存阶段发生地址错误时将错误地址赋值给BadVAddr

\item 异常处理开始时,若为TLB异常,则将错误地址高20位赋值给EntryHi高20位

\item 异常处理开始时,将Status[1]赋值为1;在执行ERET指令时将Status[1]赋值为0

\item 异常处理开始时,将Cause[6:2]赋值为异常号

\item 异常处理开始时,将EPC赋值为Victim指令地址

\item 每个周期,Count加1

\item 其他写操作由软件完成 
\end{enumerate}

\item 实现MFC0,MTC0指令访问CP0寄存器的功能。

\item 通过各寄存器值控制相应功能。

\end{enumerate}


\subsubsection{异常中断处理}
本次实验要求实现精确异常处理,以下是一些可能用到的中断及异常的情况

\begin{table}[H]
\centering
\begin{tabular}{lll}
\hline
异常号&异常名&描述\\
\hline
0&Interrupt&外部中断、异步发生,由硬件引起\\
1&TLB Modified&内存修改异常,发生在Memory阶段\\
2&TLBL&读未在TLB中映射的内存地址触发的异常\\
3&TLBS&写未在TLB中映射的内存地址触发的异常\\
4&ADEL&读访问一个非对齐地址触发的异常\\
5&ADES&读访问一个非对齐地址触发的异常\\
8&SYSCALL&系统调用\\
10&RI&执行未定义指令异常\\
11&Co-Processor Unavailable&试图访问不存在的协处理器异常\\
23&Watch&Watch寄存器监控异常\\
\hline
\end{tabular}
\end{table}

下表列出可能用到的中断号

\begin{table}[H]
\centering
\begin{tabular}{ll}
\hline
中断号&设备\\
\hline
0&系统计时器\\
1&键盘\\
3&通讯端口COM2\\
4&通讯端口COM1\\
\hline
\end{tabular}
\end{table}

中断/异常处理的一般流程如下:

\begin{enumerate}[(1)]
\item 保存中断信息,主要是EPC,BadVaddr,Status,Cause等寄存器的信息

\qquad EPC:存储异常处理之后程序恢复执行的地址。对于一般异常,当前发生错误的指令地址即为EPC应当保存的地址;而对于硬件中断,由于是异步产生则可以任意设定一条并未执行完成的指令地址保存,但在进入下一步处理之前,该指令前的指令都应当被执行完。

\qquad BadVAddr:捕捉最近一次地址错误或TLB异常(重填、失效、修改)时的虚拟地址。

\qquad Status:将EXL位置为1,进入kernel模式进行中断处理。

\qquad Cause:记录下异常号。

\qquad EntryHi:TLB异常时,记录下BadVAddr的部分高位。

\item 根据Cause中的异常号跳转到相应的异常处理函数入口

\item 中断处理

\item 通过调用ERET指令恢复现场,返回EPC所存地址执行并且将Status中的EXL重置为0表示进入user模式。

\end{enumerate}

\quad \\
实现步骤:

\begin{enumerate}[(1)]
\item 在可能发生异常的位置实现对异常的记录。

\begin{enumerate}[(a)]
\item 访存时可能发生ADEL,ADES,TLBM,TLBL,TLBS,Watch异常

\item 译码后可能发生RI,SYSCALL,Co-ProcessorU异常
\end{enumerate}
\item 实现对中断的记录。

\begin{enumerate}[(a)]
\item 硬件产生中断时将信息写入CP0寄存器
\end{enumerate}

\item 根据异常记录信息判断是否产生异常。

\item 进入异常处理流程。

\end{enumerate}

\subsubsection{MMU}
\paragraph{虚拟地址映射}
\paragraph{TLB}

\subsection{Ucore}
\subsubsection{BIOS}
BIOS即为启动ucore所用的Bootloader程序,通常是放在Flash中。而本实验中我们将在FPGA里建立一块ROM,将Bootloader放置在该ROM中,并且设置CPU的访问地址从该ROM开始。这样能避免由于Flash的读写不稳定而对BIOS造成的破坏,还能将ucore与其独立开来。

BIOS启动时,Flash中的操作系统加载到内存中,然后跳转到操作系统的初始化代码,从而开始操作系统的工作。
\subsubsection{远程文件执行}
实验要求修改ucore,实现简单的远程文件执行功能,即通过串口从PC上获取ELF文件,并在本地执行。
\subsection{外设}
\subsubsection{串口}
串口的功能需求为实现与PC机的通信,通过计算机键盘输入数据,向计算机输出数据。

串口模块的主要部分位于板子上的CPLD中,在FPGA端,对串口的控制通过data$\_$ready、tbre、\\
tsre、rdn、wrn进行。

当data$\_$ready=‘1’时串口数据就绪,可以读出。

当tbre and tsre=‘1’是,表示可以向串口写入。

将rdn置‘0’且数据线写高阻,可以从数据线得出串口数据。

将数据写入数据线且将wrn置‘0’,可以向串口发出数据。

串口模块1FD003F8地址表示数据(只有低8位有效),必须使用SW指令写入。

1FD003FC地址表示状态寄存器Status,Status$\&$1=1时可写,Status$\&$2=2时可读。
\subsubsection{VGA}
Device$\_$VGA模块是显示控制模块,接受CPU写入的ASCII码数据,并维护一个字符矩阵,以字符中断的形式通过VGA接口输出到显示器。

CPU需要从1FC03000地址读出数据,如果结果为‘1’表示可以写入,否则不能吸入;可以写入时,向1FC03000地址写入ASCII数据即可。
\subsubsection{ps/2键盘}
Device$\_$Keyboard模块位键盘控制模块。只要从0F000000地址读出数据,结果为0表示没有新数据,否则读到的就是键入的ASCII码。

PS/2键盘读到的是扫描码,为了方便软件,可以在硬件层面加入编码转换,CPU读到的直接就是ASCII码。
\subsubsection{网口}
这是拓展要求。实验提供了DM9000A网口芯片与PC机进行网络通讯。该芯片带有通用处理器接口的以太网控制器,一个10/100M PHY和4K双字的SRAM,IO端口支持3.3V与5V容限值。

当收到数据包时,芯片通过中断信号方式通知CPU触发异常,由操作系统对数据包进行处理。若要完成此需求,则需对操作系统进行改写:在初始化时添加网口中断使能、添加网口芯片初始化代码、手动实现网络通信协议和网口驱动等。
\subsection{Decaf编译器}
\subsubsection{汇编指令的生成}
由于简化的CPU中并未实现add、sub指令,需要把decaf的MIPS后端里生成add、sub指令的部分改成addu、subu,区别仅在于溢出时后者不会产生异常。

另外、CPU中也未实现除法指令,不过由于所用测试程序中没有除法运算,因此不进行相关修改。如果需要除法,可以用其它指令手动实现除法函数,并把除法翻译成函数调用。
\subsubsection{库函数调用及calling convention}
标准MIPS32使用032 ABI,函数调用的前四个参数通过\$a0-\$a3四个寄存器传输;但decaf编译出的程序的参数全都在栈上传递。当然,无论什么calling convention,只要能自恰,程序本身就应该能正常运行,所以需要解决的问题只有用于程序与C实现的库函数及操作系统交互的部分。

在这里,我们在decaf和库函数之间增加一个适配器层,将decaf的调用约定翻译成032 ABI再调用库函数。
\subsubsection{程序入口及退出}
直接使用ucore里的linker script(user.Id)以及用户静态函数库libuser.a,在该环境下系统会设置好一些全局变量,然后跳转到main执行。我们将修改decaf编译器,将其输出的main重命名为decaf$\_$main,然后汇编实现一个新的main函数。由于decaf的main是void类型,我们默认其执行成功返回0,于是在decaf$\_$main返回后直接调用exit(0)。
\subsection{指令集与数据通路}
