\section{功能需求}
\subsection{CPU}
\subsubsection{ALU}
ALU负责实现双输入的算术、逻辑和移位运算功能,其中比较运算通过补码减法实现。输入为两个32位整数和一个4位符号位,输出为32位整数,不给出标志位。

ALU需要实现的运算详见下表。

\begin{table}[ht]
\centering
\begin{tabular}{cccccc}
\hline
操作码&功能&描述&操作码&功能&描述\\
\hline
ADD&A + B&加法&NOR&$\sim$A $\mid$ B&与非\\
SUB&A - B&减法&SLL&A $\gg$ B&逻辑左移\\
AND&A $\&$ B&与&SRL&A $\ll$ B&逻辑右移\\
OR&A $\mid$ B&或&SRA&A $\ggg$ B&算术右移\\
XOR&A $\wedge$ B&异或&SLT&A $<$ B&比较\\
\hline

\end{tabular}
\end{table}

\subsubsection{乘法器}
乘法器是一个独立于ALU的元件,其中乘法运算直接使用Verilog语言提供的乘法运算符实现。输入为两个32位整数,输出为两个32位整数,分别存放在LO和HI寄存器中。

考虑到乘法运算需要的时间比较长,为尽可能简化流水线设计,可适当降低时钟频率,以使得乘法运算可以在一个时钟周期内完成运算。

\subsubsection{寄存器堆}
寄存器堆负责实现通用寄存器的读写和在数据通路中的控制,在流水线译码阶段读取一个或两个通用寄存器的数据(组合逻辑),并在流水线写回阶段将结果写入通用寄存器(时序逻辑)。

寄存器堆采用FPGA的逻辑单元来实现数据的存储,在32位Mips架构下需要实现32个32位通用寄存器。
\subsubsection{CP0}
\subsubsection{异常中断处理}
\subsubsection{MMU}
\paragraph{虚拟地址映射}
\paragraph{TLB}
\subsection{Ucore}
\subsection{外设}
\subsection{Decaf编译器}
\subsection{指令集与数据通路}
