\section{指令片段测试}
\subsection{逻辑、移位操作与空指令的测试}
具体样例见测试样例1至4。

cache指令出现译码错误情况,修改方法为将其作为特殊指令处理。

\subsection{移动操作指令的测试}
具体样例见测试样例8至9.

CPU设计时加上了延迟槽设计,故在延迟槽中放入普通指令而不是nop指令进行测试。测试结果均正确。

\subsection{算术操作指令的测试}
具体样例见测试样例7.

测试包括溢出情况。因异常处理中没有加入溢出的情况,所以当发生溢出时不做任何处理,也不进行操作。

\subsection{转移指令的测试}
具体样例见测试样例12.

测试发现mthi、mtlo指令没有正常工作,根据信号检查后发现是顶层文件中hi、lo引脚没有连,连上后测试通过。

\subsection{加载存储指令的测试}
具体样例见测试样例10

初期测试时没有考虑好结构冲突的问题,以为能将指令和数据分开存储。导致后期写仲裁器处理结构冲突时遇到了很多问题。然而在不考虑结构冲突的情况下,加载存储指令均能正确执行。

\subsection{协处理器访问指令的测试}
具体样例见测试样例12

测试均通过。

\subsection{异常相关指令的测试}
具体样例见测试样例13至17

遇见的第一个问题为应触发异常时没有正确触发异常。经过查看信号值发现原因时异常号错误。最后修改异常号后测试通过。

第二个问题时测试syscall指令时没有触发异常。经过查看信号后发现原因是进行异常处理前没有清空流水线,导致流水线仍然在工作,清空流水线后测试通过。

第三个问题时测试地址未对齐异常时没有正确触发。经过查看信号后发现原因是原先的代码在进行异常种类判定时忽略了访存阶段的异常,将访存阶段的异常加入判定后测试通过。

此外发现了诸如CPU异常和无效指令异常的bit写反了这样的小bug,更正后测试通过。
