\section{系统测试}

系统测试是片段测试的扩展,主要为bootloader和ucore的测试。

\subsection{bootloader测试}

\subsubsection{测试过程}

将bootloader的程序烧进ROM中,设定取指的起始地址为ROM的起始地址,从而执行bootloader指令。

\subsubsection{测试结果}

程序一开始就遇到了写flash的指令,导致flash被修改从而无法正确执行后面的指令。经过检查后修改了bootloader的代码,从而解决了这个问题。

之后遇到了load相关问题,发现之前对仲裁器的设计还是不合理。修改了之后load相关问题顺利解决。

再之后遇到了从flash里读出指令放入RAM时指令错误的情况,检查后发现flash与主频使用同一时钟会导致错误。修改flash工作频率为50MHz后正确运行。

至此bootloader运行通过,CPU进入ucore内核态运行。

\subsection{ucore测试}

\subsubsection{测试过程}

在执行bootloader之前先将ucore的二进制文件烧入flash中,bootloader在执行时会将指令从flash中取出来放进RAM里,再从RAM里读取指令并继续执行ucore的指令。

\subsubsection{测试结果}

在运行ucore时遇到了不存在指令异常,检查后发现是mtc0指令和mfc0指令在译码上不符合MIPS32规范。修改后执行正确

重新运行ucore时遇到了访问不存在的CP0寄存器异常。问过助教之后发现是由于我们所使用的ucore中对CP0寄存器的定义与标准MIPS32定义不同,而我们CPU中与标准MIPS32定义是相同的。这个问题至今没有解决,也说明了我们前期对ucore了解不够,导致后期失利。