\section{单元测试}
\subsection{译码模块}
\subsubsection{模块说明}

本模块用于将32位指令码翻译成具有具体功能的指令,并将对应的信号传递给ALU模块。本模块的测试重点在于能不能正确取出指令,并能正确解码指令。

\subsubsection{测试方法}

将需要测试的指令写入ROM中,然后令CPU从ROM中取出指令。然后通过LED灯查看取出的指令是否正确,并可以查看译码出来的控制信号或操作码等信号是否正确。

\subsubsection{测试样例}

所有65条指令码+1条不存在指令码

\subsubsection{测试结果}

cache指令发现了译码错误的问题,之后将其改成了特殊指令单独处理。

mtc0和mfc0在参考MIPS32指令规范时发现其指令码与实验指导文档上有出入。最后选定了规范译码方法。

\subsection{执行模块}
\subsubsection{测试方法}

将指令分类为逻辑运算指令、算数操作指令、转移指令、加载存储指令和其它指令五种类型,分别进行测试。通过查看这一阶段算出的最终结果判断是否执行正确。

\subsubsection{测试样例}

测试样例1至测试样例9,注意到了边界条件。

\subsubsection{测试结果}

所有样例可正确运行,其中mult指令可在一个周期内给出结果。

\subsection{访存模块}
\subsubsection{测试方法}

在外设没有完成的情况下用数组的方法模拟RAM,在其中进行读写。

\subsubsection{测试样例}

测试样例10

\subsubsection{测试结果}

测试发现sb指令、lb指令、lbu指令指令在处理地址后两位时候把各种情况的对应处理写反了。修改后测试通过

\subsection{CP0模块}
\subsubsection{测试方法}

使用mtc0、mfc0指令对CP0寄存器进行操作。

\subsubsection{测试样例}

测试样例12

\subsubsection{测试结果}

样例正确。

\subsection{异常处理模块}
\subsubsection{测试方法}

修改异常入口,将异常入口放在0x40的位置,主程序从0x100地址开始,从而使得主程序部分与异常处理程序部分不会发生冲突。

\subsubsection{测试样例}

测试样例14至17

\subsubsection{测试结果}

在发生异常时,PC能正确进入异常处理例程,对相应CP0寄存器进行赋值,并能正确通过eret指令返回到epc寄存器存储的地址中继续执行。

\subsection{MMU模块}
\subsubsection{模块说明}

此模块用于进行虚拟地址到物理地址的转换。

\subsubsection{测试方法}

通过加载存储指令访问各外设对应地址,看是否能选到对应外设。

\subsubsection{测试样例}

各外设一个样例,共9个样例。

\subsubsection{测试结果}

样例均正确。