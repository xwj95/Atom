\section{引言}
\subsection{编写目的}
在此前编写的需求文档中,已经明确了此次联合实验预期达到的目标,实验
中需要完成的各部分工作,也对实验中需要用到的关键技术做了简要的原理性说明,此次实验的前期准备工作的需求文档中基本体现。

进入实际的代码开发阶段,Verilog代码的编写需要更加详细的接口,更加精准的功能说明,更加细化的流程控制。从前的需求文档已经不足以对开发过程进行具体的指导了,需要一份更加详细的设计文档。

因此,为了指导代码的实际开发过程,编写此设计文档。

文档预期读者包括开发人员、任务提出者及其他需要使用该资源的用户。
\subsection{背景}
本项目的系统名称为32位MIPS处理器。

本项目任务由计算机组成原理课程刘卫东老师、李山山老师和软件工程课程白晓颖老师共同提出。

承担本项目的开发者为计33班的徐炜杰、王楠和黄欢。此外还受到了刘卫东、白晓颖、李山山三位老师的指导和王钧奕、张乐两位助教的帮助。
\subsection{开发工具}
本项目使用Xilinx ISE和Verilog HDL进行开发,使用ModelSim进行仿真,硬件系统为清华大学计算机系32位系统开发板,软件系统为ucore操作系统。
\subsection{开发流程}

\begin{enumerate}
	
	\item 明确项目需求,查阅理论资料,完成初步构想,书写需求文档。
	
	\item 根据需求文档进行设计,并对设计进行反复检验、修正,书写设计文档。
	
	\item 根据设计文档定义完成各个模块硬件代码的书写,并进行独立调试,且过程中随时对设计进行修改。
	
	\item 将硬件各模块协同联调,过程中可能对设计进行修改。
	
	\item 运行ucore操作系统,完成扩展功能。
	
\end{enumerate}
\subsection{参考资料}

\begin{enumerate}
	
	\item 实验指导文档
	
	\item OSLab实验参考文档
	
	\item 贾开, 周昕宇, 李铁铮, 等. 计算机组成原理综合实验报告
	
	\item 逆光组. 清华大学计算机组成原理32位Mips CPU教程
	
	\item 刘卫东, 李山山, 宋佳兴, 等. 计算机硬件系统实验教程[M]. 清华大学出版社, 2013.
	
	\item 帕特森, 亨尼斯, 康继昌, 等. 计算机组成与设计: 硬件/软件接口[M]. 机械工业出版社, 2012.
	
	\item Sweetman D. See MIPS run[M]. Morgan Kaufmann, 2010.
	
	\item 雷思磊. 自己动手写CPU[M]. 电子工业出版社, 2014.
	
\end{enumerate}