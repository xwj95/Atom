\section{引言}
\subsection{编写目的}
在此前编写的需求文档中,已经明确了此次联合实验预期达到的目标,实验
中需要完成的各部分工作,也对实验中需要用到的关键技术做了简要的原理性说明,此次实验的前期准备工作的需求文档中基本体现。

进入实际的代码开发阶段,Verilog代码的编写需要更加详细的接口,更加精准的功能说明,更加细化的流程控制。从前的需求文档已经不足以对开发过程进行具体的指导了,需要一份更加详细的设计文档。

因此,为了指导代码的实际开发过程,编写此设计文档。

文档预期读者包括开发人员、任务提出者及其他需要使用该资源的用户。
\subsection{背景}
本项目的系统名称为32位MIPS处理器。

本项目任务由计算机组成原理课程刘卫东老师、李山山老师和软件工程课程白晓颖老师共同提出。

承担本项目的开发者为计33班的徐炜杰、王楠和黄欢。此外还受到了刘卫东、白晓颖、李山山三位老师的指导和王钧奕、张乐两位助教的帮助。
\subsection{参考资料}
[1] 实验指导文档

[2] OSLab实验参考文档

[3] 计算机组成原理综合实验报告 贾开

[4] 刘卫东, 李山山, 宋佳兴, 等. 计算机硬件系统实验教程[M]. 清华大学出版社, 2013.

[5] 帕特森, 亨尼斯, 康继昌, 等. 计算机组成与设计: 硬件/软件接口[M]. 机械工业出版社, 2012.

[6] Sweetman D. See MIPS run[M]. Morgan Kaufmann, 2010.

[7] 雷思磊. 自己动手写CPU[M]. 电子工业出版社, 2014.