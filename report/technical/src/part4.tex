\section{总线}

为了实现CPU与外设的通信而加上了总线的设计。设计思想是CPU将选择的外设、传给外设的地址和数据输出给总线,总线将这些信号传给对应外设,外设再将CPU需要的信号通过总线传回。为了处理结构冲突,在总线上加入了仲裁器,来决定在当前处理周期进行什么操作。

\subsection{仲裁器}

总裁器是为了总线为了处理结构冲突而增加的一个部件,主要作用是确定总线在当前周期的操作应当是取指还是访存。

在优先级上,我们设定取指的优先级最高,访存读次之,访存写最次。根据该周期进行的操作,分为以下几种情况处理:

\begin{enumerate}
	\item 只进行取指操作
	
	在这种情况下不需要暂停流水线,直接执行取指即可。
	
	\item 只进行取指和访存读操作
	
	在这种情况下需要在访存阶段插入气泡,访存读行为需要等待取指行为结束后才能进行。在这个过程中流水线需暂停,直到访存读操作完成流水线才可以继续。
	
	\item 只进行取指和访存写操作
	
	在这种情况下需要在访存阶段插入气泡,访存写行为需要等待取指行为结束后才能进行。在这个过程中流水线需暂停,直到访存写操作完成流水线才可以继续。
	
	\item 需进行取指、访存读和访存写操作
	
	在这种情况下需要在访存阶段插入气泡,使得取指、访存读、访存写行为能够依次执行。在这个过程中流水线需暂停,直到访存写操作完成流水线才可以继续。
	
	\item 仲裁器收到了来自CPU的暂停请求
	
	这种情况是由于CPU遇到了load相关问题,需要令数据前推而产生的。load相关问题是指CPU在load类指令后遇到了跳转指令从而导致的数据冲突问题。在CPU的模块中,我们为了处理这个冲突,会在流水线中插入一个气泡,从而使跳转指令的译码晚一个周期,令load类指令能完成访存,从而在访存阶段得以将数据前推回译码阶段。
	
	对仲裁器而言,当遇到来自CPU的暂停请求时,就意味着仲裁器需要跳过取指阶段,直接执行访存读。
		
\end{enumerate}

\subsection{外设处理}

选定外设后,总线给出地址、数据和读写使能,外设则将数据写入对应地址或从对应地址取出数据传回总线。当读/写操作完成后会给出ack信号,传回仲裁器中。对仲裁器而言,如果没有来自外设的ack信号,便会暂停流水线,直到收到ack信号为止。

以下是外设的说明。

\begin{enumerate}
	\item RAM
	
	为了加快从RAM里取指的速度,将RAM的读操作压缩在一个周期内完成。实验证明在25M时钟下可正确运行,在50M时钟下会取指错误。RAM写则是两个周期。
	
	RAM可读可写。
	
	\item ROM
	
	ROM只用于取指令,里面的指令直接写在代码里。当需要从中取指时,只需一个周期便可从对应地址取出指令。在运行ucore的时候ROM里放的是bootloader的指令。
	
	ROM只可读不可写。
	
	\item FLASH
	
	由于FLASH断电可保存信息,所以提前将ucore的内容写进FLASH中。在bootloader时,CPU将存放在FLASH里的ucore指令取出,写进RAM中。由于FLASH读操作较慢,所以不使用CPU主频,而是接50M时钟。在50M时钟下,FLASH读需要八个周期。
	
	FLASH只可读不可写。
	
	\item UART
	
	我们参考了fpga4fun中处理串口的代码。串口也不使用CPU主频,而是接50M时钟。
	
	UART可读可写。
	
\end{enumerate}
