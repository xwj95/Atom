\section{模块设计}
\subsection{取指阶段}
取指阶段取出指令寄存器中的指令,同时,PC递增,准备取下一条指令,包括PC、IF/ID两个模块。
\subsubsection{PC模块}
\paragraph{功能描述}
\quad

\quad

给出指令地址,其中实现指令指针寄存器PC,该寄存器的值就是指令地址,对应pc$\_$reg.v文件。

PC的取值有三种情况:
\begin{enumerate}[(1)]
	\item 一般情况下,PC等于PC+4,每个时钟周期PC加4,指向下一条指令。
	
	\item 当流水线暂停的时候,PC保持不变。
	
	\item 如果是转移指令,且满足转移条件,那么将转移目标地址赋给PC,即PC等于转移判断的结果。
\end{enumerate}
\paragraph{端口说明}
\quad

\quad
\begin{table}[H]
	\centering
	\caption{PC模块的接口描述}
	\begin{tabular}{|l|l|l|l|}
		\hline
		接口名 & 宽度(bit) & 输入/输出 & 作用 \\
		\hline
		rst & 1 & 输入 & 复位信号 \\
		\hline
		clk & 1 & 输入 & 时钟信号 \\
		\hline
		branch$\_$flag$\_$i & 1 & 输入 & 是否发生转移 \\
		\hline
		branch$\_$target$\_$address$\_$i & 32 & 输入 & 转移到的目标地址 \\
		\hline
		flush & 1 & 输入 & 流水线清除信号 \\
		\hline
		new$\_$pc & 32 & 输入 & 异常处理例程入口地址 \\
		\hline
		stall & 1 & 输入 & 取指地址PC是否保持不变 \\
		\hline
		pc & 32 & 输出 & 要读取的指令地址 \\
		\hline
		ce & 1 & 输出 & 指令存储器ROM使能信号 \\
		\hline
	\end{tabular}
\end{table}
\subsubsection{IF/ID模块}
\paragraph{功能描述}
\quad

\quad

实现取指与译码阶段之间的寄存器,将取指阶段的结果(取得的指令、指令地址等信息)暂时保存,在下一个时钟周期的上升沿传递到译码阶段,对应if$\_$id.v文件。其实现只有一个时序逻辑电路。
\paragraph{端口说明}
\quad

\quad
\begin{table}[H]
	\centering
	\caption{IF/ID模块的接口描述}
	\begin{tabular}{|l|l|l|l|}
		\hline
		接口名 & 宽度(bit) & 输入/输出 & 作用 \\
		\hline
		rst & 1 & 输入 & 复位信号 \\
		\hline
		clk & 1 & 输入 & 时钟信号 \\
		\hline
		if$\_$pc & 32 & 输入 & 取指阶段取出的指令对应的地址 \\
		\hline
		if$\_$inst & 32 & 输入 & 取指阶段取出的指令 \\
		\hline
		stall & 1 & 输入 & 取指阶段是否暂停 \\
		\hline
		flush & 1 & 输入 & 流水线清除信号 \\
		\hline
		id$\_$pc & 32 & 输出 & 译码阶段的指令对应的地址 \\
		\hline
		id$\_$inst & 32 & 输出 & 译码阶段的指令 \\
		\hline
	\end{tabular}
\end{table}
\subsection{译码阶段}
译码阶段将对取到的指令进行译码:给出要进行的运算类型,以及参与运算的操作数。译码阶段包括Regfile、ID和ID/EX三个模块。
\subsubsection{Regfile模块}
\paragraph{功能描述}
\quad

\quad

实现了32个32位通用整数寄存器,可以同时进行两个寄存器的读操作和一个寄存器的写操作,对应regfile.v文件。

Regfile模块可以分为四段进行理解。
\begin{enumerate}[(1)]
	\item 第一段:定义了一个二维的向量,为32个32位寄存器。
	
	\item 第二段:实现了写寄存器操作,当复位信号无效时(rst为RstDisable),在写使能信号we有效(we为WriteEnable),且写操作目的寄存器不等于0的情况下,可以将写输入数据保存到目的寄存器。之所以要判断目的寄存器不为0,是因为MIPS32架构规定$\$$0的值只能为0,所以不要写入。
	
	\item 第三段:实现了第一个读寄存器端口,分以下几步依次判断:
	
	\begin{enumerate}[(a)]
		\item 当复位信号有效时,第一个读寄存器端口的输出始终为0;
		
		\item 当复位信号无效时,如果读取的是$\$$0,那么直接给出0;
		
		\item 如果第一个读寄存器端口要读取的目标寄存器与要写入的目的寄存器是同一个寄存器,那么直接将要写入的值作为第一个读寄存器端口的输出;
		
		\item 上述情况都不满足,那么给出第一个读寄存器端口要读取的目标寄存器地址对应寄存器的值;
		
		\item 第一个读寄存器端口没有使能时,直接输出0。
	\end{enumerate}
	
	\item 第四段:实现了第二个读寄存器端口,具体过程与第三段是相似的,不再重复解释。
\end{enumerate}

注意一点:读寄存器操作是组合逻辑电路,也就是一旦输入的要读取的寄存器地址raddr1或者raddr2发生变化,那么会立即给出新地址对应的寄存器的值,这样可以保证在译码阶段取得要读取的寄存器的值,而写寄存器操作是时序逻辑电路,写操作发生在时钟信号的上升沿。

\paragraph{端口说明}
\quad

\quad
\begin{table}[H]
	\centering
	\caption{Regfile模块的接口描述}
	\begin{tabular}{|l|l|l|l|}
		\hline
		接口名 & 宽度(bit) & 输入/输出 & 作用 \\
		\hline
		rst & 1 & 输入 & 复位信号,高电平有效 \\
		\hline
		clk & 1 & 输入 & 时钟信号 \\
		\hline
		waddr & 5 & 输入 & 要写入的寄存器地址 \\
		\hline
		wdata & 32 & 输入 & 要写入的数据 \\
		\hline
		we & 1 & 输入 & 写使能信号 \\
		\hline
		raddr1 & 5 & 输入 & 第一个读寄存器端口要读取的寄存器的地址 \\
		\hline
		re1 & 1 & 输入 & 第一个读寄存器端口读使能信号 \\
		\hline
		rdata1 & 32 & 输出 & 第一个读寄存器端口输出的寄存器值 \\
		\hline
		raddr2 & 5 & 输入 & 第二个读寄存器端口要读取的寄存器的地址 \\
		\hline
		re2 & 1 & 输入 & 第二个读寄存器端口读使能信号 \\
		\hline
		rdata2 & 32 & 输出 & 第二个读寄存器端口输出的寄存器值 \\
		\hline
	\end{tabular}
\end{table}
\subsubsection{ID模块}
\paragraph{功能描述}
\quad

\quad

对指令进行译码,译码结果包括运算类型、运算所需的源操作数、要写入的目的寄存器地址等,对应id.v文件。其中,运算类型指的是逻辑运算、移位运算、算术运算等,子类型指的是更详细的运算类型,比如运算类型为逻辑运算时,运算子类型可以是逻辑“或”运算、逻辑“与”运算、逻辑“异或”运算等。

ID模块中的电路都是组合逻辑电路,ID模块与Regfile模块之间有接口连接。其实现可以分为三段进行理解:

\begin{enumerate}[(1)]
	\item 第一段:实现了对指令的译码,依据指令中的特征字段区分指令。以指令ori为例:只需通过识别26-31bit的指令码是否是6'b001101,即可判断是否是ori指令,其中的宏定义EXE$\_$ORI就是6'b001101,op就是指令的26-31bit,所以当op等于EXE$\_$ORI时,就表示是ori指令,此时会有以下译码结果。
	
	\begin{enumerate}[(a)]
		\item 要读取的寄存器情况:ori指令只需要读取rs寄存器的值,默认通过Regfile读端口1读取的寄存器地址reg1$\_$addr$\_$o的值是指令的21-25bit,正是ori指令中的rs,所以设置reg1$\_$read$\_$o为1,reg1$\_$read$\_$o 连接Regfile的输入re1,reg1$\_$addr$\_$o连接Regfile的输入raddr1,结合对Regfile模块的介绍可知,译码阶段会读取寄存器rs的值。指令ori需要的另一个操作数是立即数,所以设置reg2$\_$read$\_$o为0,表示不通过Regfile读端口2读取寄存器,这里暗含使用立即数作为运算的操作数。imm就是指令中的立即数进行零扩展后的值。
		
		\item 要执行的运算:alusel$\_$o给出要执行的运算类型,对于ori指令而言就是逻辑操作,即EXE$\_$RES$\_$LOGIC。 aluop$\_$o给出要执行的运算子类型,对于ori指令而言就是逻辑“或”运算,即EXE$\_$OR$\_$OP。这两个值会传递到执行阶段。
		
		\item 要写入的目的寄存器:wreg$\_$o表示是否要写目的寄存器,ori指令要将计算结果保存到寄存器中,所以wreg$\_$o设置为WriteEnable。wd$\_$o是要写入的目的寄存器地址,此时就是指令的16-20bit,正是ori指令中的rt。这两个值也会传递到执行阶段。
	\end{enumerate}
	
	\item 第二段:给出参与运算的源操作数1的值,如果reg1$\_$read$\_$o为1,那么就将从Regfile模块读端口1读取的寄存器的值作为源操作数1,如果reg1$\_$read$\_$o为0,那么就将立即数作为源操作数1,对于ori而言,此处选择从Regfile模块读端口1读取的寄存器的值作为源操作数1。该值将通过reg1$\_$o端口被传递到执行阶段。
	
	\item 第三段:给出参与运算的源操作数2的值,如果reg2$\_$read$\_$o为1,那么就将从Regfile模块读端口2读取的寄存器的值作为源操作数2,如果reg2$\_$read$\_$o为0,那么就将立即数作为源操作数2,对于ori而言,此处选择立即数imm作为源操作数2。该值将通过reg2$\_$o端口被传递到执行阶段。
\end{enumerate}

ID模块还处理了冲突相关的部分内容。
\begin{enumerate}
	\item 实现数据前推的方法具体如下:
	\begin{enumerate}[(1)]
		\item 将处于流水线执行阶段的指令的运算结果,包括:是否要写目的寄存器wreg$\_$o、要写的目的寄存器地址wd$\_$o、要写入目的寄存器的数据wdata$\_$o等信息送到译码阶段。
	
		\item 将处于流水线访存阶段的指令的运算结果,包括:是否要写目的寄存器wreg$\_$o、要写的目的寄存器地址wd$\_$o、要写入目的寄存器的数据wdata$\_$o等信息送到译码阶段。
	\end{enumerate}
	\item load相关
	\begin{enumerate}[(1)]
		\item 即使通过数据前推的方法,将访存阶段加载得到的数据前推,也解决不了问题,因为数据加载时,下一条指令如果是beq指令,就已经处于执行阶段了,已经进行了比较判断。
		
		\item OpenMIPS解决load相关的方法是,在译码阶段检查当前指令与上一条指令是否存在load相关,如果存在load相关,那么就让流水线的译码、取指阶段暂停,而执行、访存、回写阶段继续,相当于插入一个空指令,这样处于执行阶段的加载指令会继续运行,不受影响,当其运行到访存阶段时,将加载得到的数据前推到译码阶段,然后,流水线可以继续运行。
		
		\item 具体地,如果存在load相关,那么通过stallreq接口通知CTRL模块请求流水线暂停。
	\end{enumerate}
\end{enumerate}


此外,有关转移指令的实现,如果在流水线执行阶段进行转移判断,并且发生转移,那么就会有2条无效指令,导致浪费了两个时钟周期。为了减少损失,规定转移指令后面的指令位置为“延迟槽”,延迟槽中的指令被称为“延迟指令”。延迟指令总是被执行,与转移发生与否没有关系。为了避免时钟周期的浪费,我们在译码阶段进行了转移判断:

\begin{enumerate}[(1)]
	\item 如果处于译码阶段的指令是转移指令,并且满足转移条件,那么ID模块设置转移发生标志branch$\_$flag$\_$o为Branch,同时通过branch$\_$target$\_$address$\_$o接口给出转移目的地址,送到PC模块,后者据此修改取指地址。
	
	\item 如果处于译码阶段的指令是转移指令,并且满足转移条件,那么ID模块还会设置next$\_$inst$\_$in$\_$delayslot$\_$o为InDelaySlot,表示下一条指令是延迟槽指令,其中InDelaySlot是一个宏定义。next$\_$inst$\_$in$\_$delayslot$\_$o信号会送入ID/EX模块,并在下一个时钟周期通过ID/EX模块的is$\_$in$\_$delayslot$\_$o接口送回到ID模块,ID模块可以据此判断当前处于译码阶段的指令是否是延迟槽指令。
	
	\item 如果转移指令需要保存返回地址,那么ID模块还要计算返回地址,并通过link$\_$addr$\_$o接口输出,该值最终会传递到EX模块,作为要写入目的寄存器的值。
\end{enumerate}

\paragraph{端口说明}
\quad

\quad
\begin{table}[H]
	\centering
	\caption{ID模块的接口描述}
	\begin{tabular}{|l|l|l|l|}
		\hline
		接口名 & 宽度(bit) & 输入/输出 & 作用 \\
		\hline
		rst & 1 & 输入 & 复位信号 \\
		\hline
		pc$\_$i & 32 & 输入 & 译码阶段的指令对应的地址 \\
		\hline
		inst$\_$i & 32 & 输入 & 译码阶段的指令 \\
		\hline
		reg1$\_$data$\_$i & 32 & 输入 & 从Regfile输入的第一个读寄存器端口的输入 \\
		\hline
		reg2$\_$data$\_$i & 32 & 输入 & 从Regfile输入的第二个读寄存器端口的输入 \\
		\hline
		ex$\_$wreg$\_$i & 1 & 输入 & 处于执行阶段的指令是否要写目的寄存器 \\
		\hline
		ex$\_$wd$\_$i & 5 & 输入 & 处于执行阶段的指令要写的目的寄存器地址 \\
		\hline
		ex$\_$wdata$\_$i & 32 & 输入 & 处于执行阶段的指令要写入目的寄存器的数据 \\
		\hline
		mem$\_$wreg$\_$i & 1 & 输入 & 处于访存阶段的指令是否要写目的寄存器 \\
		\hline
		mem$\_$wd$\_$i & 5 & 输入 & 处于访存阶段的指令要写的目的寄存器地址 \\
		\hline
		mem$\_$wdata$\_$i & 32 & 输入 & 处于访存阶段的指令要写入目的寄存器的数据 \\
		\hline
		ex$\_$aluop$\_$i & 8 & 输入 & 处于执行阶段指令的运算子类型 \\
		\hline
		is$\_$in$\_$delayslot$\_$i & 1 & 输入 & 当前处于译码阶段的指令是否位于延迟槽 \\
		\hline
		reg1$\_$read$\_$o & 1 & 输出 & Regfile模块的第一个读寄存器端口的读使能信号 \\
		\hline
		reg2$\_$read$\_$o & 1 & 输出 & Regfile模块的第二个读寄存器端口的读使能信号 \\
		\hline
		reg1$\_$addr$\_$o & 5 & 输出 & Regfile模块的第一个读寄存器端口的读地址信号 \\
		\hline
		reg2$\_$addr$\_$o & 5 & 输出 & Regfile模块的第二个读寄存器端口的读地址信号 \\
		\hline
		aluop$\_$o & 8 & 输出 & 译码阶段的指令要进行的运算的子类型 \\
		\hline
		alusel$\_$o & 3 & 输出 & 译码阶段的指令要进行的运算的类型 \\
		\hline
		reg1$\_$o & 32 & 输出 & 译码阶段的指令要进行的运算的源操作数1 \\
		\hline
		reg2$\_$o & 32 & 输出 & 译码阶段的指令要进行的运算的源操作数2 \\
		\hline
		wd$\_$o & 5 & 输出 & 译码阶段的指令要写入的目的寄存器地址 \\
		\hline
		wreg$\_$o & 1 & 输出 & 译码阶段的指令是否有要写入的目的寄存器 \\
		\hline
		branch$\_$flag$\_$o & 1 & 输出 & 是否发生转移 \\
		\hline
		branch$\_$target$\_$address$\_$o & 32 & 输出 & 转移到的目标地址 \\
		\hline
		is$\_$in$\_$delayslot$\_$o & 1 & 输出 & 当前处于译码阶段的指令是否位于延迟槽 \\
		\hline
		link$\_$addr$\_$o & 32 & 输出 & 转移指令要保存的返回地址 \\
		\hline
		next$\_$inst$\_$in$\_$delayslot$\_$o & 1 & 输出 & 下一条进入译码阶段的指令是否位于延迟槽 \\
		\hline
		inst$\_$o & 32 & 输出 & 当前处于译码阶段的指令 \\
		\hline
		excepttype$\_$o & 32 & 输出 & 收集的异常信息 \\
		\hline
		current$\_$inst$\_$addr$\_$o & 32 & 输出 & 译码阶段指令的地址 \\
		\hline
		stallreq & 1 & 输出 & 译码阶段请求流水暂停 \\
		\hline
	\end{tabular}
\end{table}
\subsubsection{ID/EX模块}
\paragraph{功能描述}
\quad

\quad

实现译码与执行阶段之间的寄存器,将译码阶段取得的运算类型、源操作数、要写的目的寄存器地址等结果暂时保存,在下一个时钟周期的上升沿传递到流水线的执行阶段,对应id$\_$ex.v文件。其实现只有一个时序逻辑电路。
\paragraph{端口说明}
\quad

\quad
\begin{table}[H]
	\centering
	\caption{ID/EX模块的接口描述}
	\begin{tabular}{|l|l|l|l|}
		\hline
		接口名 & 宽度(bit) & 输入/输出 & 作用 \\
		\hline
		rst & 1 & 输入 & 复位信号 \\
		\hline
		clk & 1 & 输入 & 时钟信号 \\
		\hline
		id$\_$alusel & 3 & 输入 & 译码阶段的指令要进行的运算的类型 \\
		\hline
		id$\_$aluop & 8 & 输入 & 译码阶段的指令要进行的运算的子类型 \\
		\hline
		id$\_$reg1 & 32 & 输入 & 译码阶段的指令要进行的运算的源操作数1 \\
		\hline
		id$\_$reg2 & 32 & 输入 & 译码阶段的指令要进行的运算的源操作数2 \\
		\hline
		id$\_$wd & 5 & 输入 & 译码阶段的指令要写入的目的寄存器地址 \\
		\hline
		id$\_$wreg & 1 & 输入 & 译码阶段的指令是否有要写入的目的寄存器 \\
		\hline
		stall & 1 & 输入 & 译码阶段是否暂停 \\
		\hline
		flush & 1 & 输入 & 流水线清除信号 \\
		\hline
		id$\_$excepttype & 32 & 输入 & 译码阶段收集到的异常信息 \\
		\hline
		id$\_$current$\_$inst$\_$addr & 32 & 输入 & 译码阶段指令的地址 \\
		\hline
		id$\_$is$\_$in$\_$delayslot & 1 & 输入 & 当前处于译码阶段的指令是否位于延迟槽 \\
		\hline
		id$\_$link$\_$address & 32 & 输入 & 处于译码阶段的转移指令要保存的返回地址 \\
		\hline
		next$\_$inst$\_$in$\_$delayslot$\_$i & 1 & 输入 & 下一条进入译码阶段的指令是否位于延迟槽 \\
		\hline
		id$\_$inst & 32 & 输入 & 当前处于译码阶段的指令 \\
		\hline
		ex$\_$inst & 32 & 输出 & 当前处于执行阶段的指令 \\
		\hline
		ex$\_$is$\_$in$\_$delayslot & 32 & 输出 & 当前处于执行阶段的指令是否位于延迟槽 \\
		\hline
		ex$\_$link$\_$address & 1 & 输出 & 处于执行阶段的转移指令要保存的返回地址 \\
		\hline
		is$\_$in$\_$delayslot$\_$o & 1 & 输出 & 当前处于译码阶段的指令是否位于延迟槽 \\
		\hline
		ex$\_$excepttype & 32 & 输出 & 译码阶段收集到的异常信息 \\
		\hline
		ex$\_$current$\_$inst$\_$addr & 32 & 输出 & 执行阶段指令的地址 \\
		\hline
		ex$\_$alusel & 3 & 输出 & 执行阶段的指令要进行的运算的类型 \\
		\hline
		ex$\_$aluop & 8 & 输出 & 执行阶段的指令要进行的运算的子类型 \\
		\hline
		ex$\_$reg1 & 32 & 输出 & 执行阶段的指令要进行的运算的源操作数1 \\
		\hline
		ex$\_$reg2 & 32 & 输出 & 执行阶段的指令要进行的运算的源操作数2 \\
		\hline
		ex$\_$wd & 5 & 输出 & 执行阶段的指令要写入的目的寄存器地址 \\
		\hline
		ex$\_$wreg & 1 & 输出 & 执行阶段的指令是否有要写入的目的寄存器 \\
		\hline
	\end{tabular}
\end{table}
\subsection{执行阶段}
执行阶段将依据译码阶段的结果,对源操作数1、源操作数2进行指定的运算。执行阶段包括EX、EX/MEM两个模块。
\subsubsection{EX模块}
\paragraph{功能描述}
\quad

\quad

依据接口关系,EX模块会从ID/EX模块得到运算类型alusel$\_$i、运算子类型aluop$\_$i、源操作数reg1$\_$i、源操作数reg2$\_$i、要写的目的寄存器地址wd$\_$i。依据这些译码阶段的结果,进行指定的运算,给出运算结果。对应ex.v文件。

EX模块中都是组合逻辑电路,其实现可以分为两段理解:
\begin{enumerate}[(1)]
	\item 第一段:依据输入的运算子类型进行运算,比如逻辑“或”运算。运算结果根据不同的运算类型进行保存,如逻辑操作的结果保存在logicout中,算术运算的结果保存在arithmeticres中,移位运算的结果保存在shiftres中等。
	
	\item 第二段:给出最终的运算结果,包括:是否要写目的寄存器wreg$\_$o、要写的目的寄存器地址wd$\_$o、要写入的数据wdata$\_$o。其中wreg$\_$o、wd$\_$o的值都直接来自译码阶段,不需要改变,wdata$\_$o的值要依据运算类型进行选择,如果是逻辑运算,那么将logicout的值赋给wdata$\_$o。如果是mthi、mtlo指令,为了写HI、LO寄存器,还要给出whilo$\_$o、hi$\_$o、lo$\_$o的值。
\end{enumerate}

EX模块还处理了冲突相关的部分内容,具体如下。

实现数据前推的方法:将处于流水线访存阶段、回写阶段的指令对HI、LO寄存器的操作信息反馈到执行阶段,如果处于执行阶段的是mfhi、mflo指令,执行阶段的选择模块就会从中选择,确定HI、LO寄存器的正确值。
\paragraph{端口说明}
\quad

\quad
\begin{table}[H]
	\centering
	\caption{EX模块的接口描述}
	\begin{tabular}{|l|l|l|l|}
		\hline
		接口名 & 宽度(bit) & 输入/输出 & 作用 \\
		\hline
		rst & 1 & 输入 & 复位信号 \\
		\hline
		alusel$\_$i & 3 & 输入 & 执行阶段要进行的运算的类型 \\
		\hline
		aluop$\_$i & 8 & 输入 & 执行阶段要进行的运算的子类型 \\
		\hline
		reg1$\_$i & 32 & 输入 & 参与运算的源操作数1 \\
		\hline
		reg2$\_$i & 32 & 输入 & 参与运算的源操作数2 \\
		\hline
		wd$\_$i & 5 & 输入 & 指令执行要写入的目的寄存器地址 \\
		\hline
		wreg$\_$i & 1 & 输入 & 是否有要写入的目的寄存器 \\
		\hline
		excepttype$\_$i & 32 & 输入 & 译码阶段收集到的异常信息 \\
		\hline
		current$\_$inst$\_$addr$\_$i & 32 & 输入 & 执行阶段指令的地址 \\
		\hline
		is$\_$in$\_$delayslot$\_$i & 1 & 输入 & 当前处于执行阶段的指令是否位于延迟槽 \\
		\hline
		link$\_$address$\_$i & 32 & 输入 & 处于执行阶段的转移指令要保存的返回地址 \\
		\hline
		hilo$\_$temp$\_$i & 64 & 输入 & 第一个执行周期得到的乘法结果 \\
		\hline
		cnt$\_$i & 2 & 输入 & 当前处于执行阶段的第几个时钟周期 \\
		\hline
		hilo$\_$temp$\_$o & 64 & 输出 & 第一个执行周期得到的乘法结果 \\
		\hline
		cnt$\_$o & 2 & 输出 & 下一个时钟周期处于执行阶段的第几个时钟周期 \\
		\hline
		excepttype$\_$o & 32 & 输出 & 译码阶段、执行阶段搜集到的异常信息 \\
		\hline
		current$\_$inst$\_$addr$\_$o & 32 & 输出 & 执行阶段指令的地址 \\
		\hline
		is$\_$in$\_$delayslot$\_$o & 1 & 输出 & 执行阶段的指令是否是延迟槽指令 \\
		\hline
		wd$\_$o & 5 & 输出 & 执行阶段的指令最终要写入的目的寄存器地址 \\
		\hline
		wreg$\_$o & 1 & 输出 & 执行阶段的指令最终是否有要写入的目的寄存器 \\
		\hline
		wdata$\_$o & 32 & 输出 & 执行阶段的指令最终要写入目的寄存器的值 \\
		\hline
		hi$\_$i & 32 & 输入 & HILO模块给出的HI寄存器的值 \\
		\hline
		lo$\_$i & 32 & 输入 & HILO模块给出的LO寄存器的值 \\
		\hline
		mem$\_$whilo$\_$i & 1 & 输入 & 处于访存阶段的指令是否要写HI、LO寄存器 \\
		\hline
		mem$\_$hi$\_$i & 32 & 输入 & 处于访存阶段的指令要写入HI寄存器的值 \\
		\hline
		mem$\_$lo$\_$i & 32 & 输入 & 处于访存阶段的指令要写入LO寄存器的值 \\
		\hline
		wb$\_$whilo$\_$i & 1 & 输入 & 处于写回阶段的指令是否要写HI、LO寄存器 \\
		\hline
		wb$\_$hi$\_$i & 32 & 输入 & 处于写回阶段的指令要写入HI寄存器的值 \\
		\hline
		wb$\_$lo$\_$i & 32 & 输入 & 处于写回阶段的指令要写入LO寄存器的值 \\
		\hline
	\end{tabular}
\end{table}
\begin{table}[H]
	\centering
	\begin{tabular}{|l|l|l|l|}
		\hline
		whilo$\_$o & 1 & 输出 & 处于执行阶段的指令是否要写HI、LO寄存器 \\
		\hline
		hi$\_$o & 32 & 输出 & 处于执行阶段的指令要写入HI寄存器的值 \\
		\hline
		lo$\_$o & 32 & 输出 & 处于执行阶段的指令要写入LO寄存器的值 \\
		\hline
		cp0$\_$reg$\_$data$\_$i & 32 & 输入 & 从CP0模块读取的指定寄存器的值 \\
		\hline
		mem$\_$cp0$\_$reg$\_$we & 1 & 输入 & 访存阶段的指令是否要写CP0中的寄存器 \\
		\hline
		mem$\_$cp0$\_$reg$\_$write$\_$addr & 5 & 输入 & 访存阶段的指令要写的CP0中寄存器的地址 \\
		\hline
		mem$\_$cp0$\_$reg$\_$data & 32 & 输入 & 访存阶段的指令要写入CP0中寄存器的数据 \\
		\hline
		wb$\_$cp0$\_$reg$\_$we & 1 & 输入 & 写回阶段的指令是否要写CP0中的寄存器 \\
		\hline
		wb$\_$cp0$\_$reg$\_$write$\_$addr & 5 & 输入 & 写回阶段的指令要写的CP0中寄存器的地址 \\
		\hline
		wb$\_$cp0$\_$reg$\_$data & 32 & 输入 & 写回阶段的指令要写入CP0中寄存器的数据 \\
		\hline
		cp0$\_$reg$\_$read$\_$addr$\_$o & 5 & 输出 & 执行阶段的指令要读取的CP0中寄存器的地址 \\
		\hline
		cp0$\_$reg$\_$we$\_$o & 1 & 输出 & 执行阶段的指令是否要写CP0中的寄存器 \\
		\hline
		cp0$\_$reg$\_$write$\_$addr$\_$o & 5 & 输出 & 执行阶段的指令要写的CP0中寄存器的地址 \\
		\hline
		cp0$\_$reg$\_$data$\_$o & 32 & 输出 & 执行阶段的指令要写入CP0中寄存器的数据 \\
		\hline
		inst$\_$i & 32 & 输入 & 当前处于执行阶段的指令 \\
		\hline
		aluop$\_$o & 8 & 输出 & 执行阶段的指令要进行的运算的子类型 \\
		\hline
		mem$\_$addr$\_$o & 32 & 输出 & 加载、存储指令对应的存储器地址 \\
		\hline
		\multirow{2}{*}{reg2$\_$o$\_$pc} & \multirow{2}{*}{32} & \multirow{2}{*}{输出} & 存储指令要存储的数据,或者lwl、lwr指令 \\
		& & & 要写入的目的寄存器的原始值 \\
		\hline
		stallreq & 1 & 输出 & 执行阶段是否请求流水线暂停 \\
		\hline
	\end{tabular}
\end{table}
\subsubsection{EX/MEM模块}
\paragraph{功能描述}
\quad

\quad

实现执行与访存阶段之间的寄存器,将执行阶段的结果暂时保存,在下一个时钟周期的上升沿传递到流水线的访存阶段,对应ex$\_$mem.v文件。其实现只有一个时序逻辑电路。
\paragraph{端口说明}
\quad

\quad
\begin{table}[H]
	\centering
	\caption{EX/MEM模块的接口描述}
	\begin{tabular}{|l|l|l|l|}
		\hline
		接口名 & 宽度(bit) & 输入/输出 & 作用 \\
		\hline
		rst & 1 & 输入 & 复位信号 \\
		\hline
		clk & 1 & 输入 & 时钟信号 \\
		\hline
		ex$\_$wd & 5 & 输入 & 执行阶段的指令执行后要写入的目的寄存器地址 \\
		\hline
		ex$\_$wreg & 1 & 输入 & 执行阶段的指令执行后是否有要写入的目的寄存器 \\
		\hline
		ex$\_$wdata & 32 & 输入 & 执行阶段的指令执行后要写入的目的寄存器的值 \\
		\hline
		mem$\_$wd & 5 & 输出 & 访存阶段的指令要写入的目的寄存器地址 \\
		\hline
		mem$\_$wreg & 1 & 输出 & 访存阶段的指令是否有要写入的目的寄存器 \\
		\hline
		mem$\_$wdata & 32 & 输出 & 访存阶段的指令要写入目的寄存器的值 \\
		\hline
		stall & 1 & 输入 & 执行阶段是否暂停 \\
		\hline
		flush & 1 & 输入 & 是否清除流水线 \\
		\hline
		ex$\_$cp0$\_$reg$\_$we & 1 & 输入 & 执行阶段的指令是否要写CP0中的寄存器 \\
		\hline
		ex$\_$cp0$\_$reg$\_$write$\_$addr & 5 & 输入 & 执行阶段的指令要写的CP0中寄存器的地址 \\
		\hline
		ex$\_$cp0$\_$reg$\_$data & 32 & 输入 & 执行阶段的指令要写入CP0中寄存器的数据 \\
		\hline
		mem$\_$cp0$\_$reg$\_$we & 1 & 输出 & 访存阶段的指令是否要写CP0中的寄存器 \\
		\hline
		mem$\_$cp0$\_$reg$\_$write$\_$addr & 5 & 输出 & 访存阶段的指令要写的CP0中寄存器的地址 \\
		\hline
		mem$\_$cp0$\_$reg$\_$data & 32 & 输出 & 访存阶段的指令要写入CP0中寄存器的数据 \\
		\hline
		ex$\_$aluop & 8 & 输入 & 执行阶段的指令要进行的运算的子类型 \\
		\hline
		ex$\_$mem$\_$addr & 32 & 输入 & 执行阶段的加载、存储指令对应的存储器地址 \\
		\hline
		\multirow{2}{*}{ex$\_$reg2} & \multirow{2}{*}{32} & \multirow{2}{*}{输入} & 执行阶段的存储指令要存储的数据,或者lwl、lwr \\
		& & & 指令要写入的目的寄存器的原始值 \\
		\hline
		mem$\_$aluop & 8 & 输出 & 访存阶段的指令要进行的运算的子类型 \\
		\hline
		mem$\_$mem$\_$addr & 32 & 输出 & 访存阶段的加载、存储指令对应的存储器地址 \\
		\hline
		\multirow{2}{*}{mem$\_$reg2} & \multirow{2}{*}{32} & \multirow{2}{*}{输出} & 访存阶段的存储指令要存储的数据,或者lwl、lwr \\
		& & & 指令要写入的目的寄存器的原始值 \\
		\hline
		ex$\_$whilo & 1 & 输入 & 执行阶段的指令是否要写HI、LO寄存器 \\
		\hline
		ex$\_$hi & 32 & 输入 & 执行阶段的指令要写入HI寄存器的值 \\
		\hline
		ex$\_$lo & 32 & 输入 & 执行阶段的指令要写入LO寄存器的值 \\
		\hline
		mem$\_$whilo & 1 & 输出 & 访存阶段的指令是否要写HI、LO寄存器 \\
		\hline
		mem$\_$hi & 32 & 输出 & 访存阶段的指令要写入HI寄存器的值 \\
		\hline
		mem$\_$lo & 32 & 输出 & 访存阶段的指令要写入LO寄存器的值 \\
		\hline
	\end{tabular}
\end{table}
\begin{table}[H]
	\centering
	\begin{tabular}{|l|l|l|l|}
		\hline
		ex$\_$excepttype & 32 & 输入 & 译码、执行阶段收集到的异常信息 \\
		\hline
		ex$\_$current$\_$inst$\_$address & 32 & 输入 & 执行阶段指令的地址 \\
		\hline
		ex$\_$is$\_$in$\_$delayslot & 1 & 输入 & 执行阶段的指令是否是延迟槽指令 \\
		\hline
		mem$\_$excepttype & 32 & 输出 & 译码、执行阶段收集到的异常信息 \\
		\hline
		mem$\_$current$\_$inst$\_$address & 32 & 输出 & 访存阶段指令的地址 \\
		\hline
		mem$\_$is$\_$in$\_$delayslot & 1 & 输出 & 访存阶段的指令是否是延迟槽指令 \\
		\hline
		hilo$\_$i & 64 & 输入 & 保存的乘法结果 \\
		\hline
		cnt$\_$i & 2 & 输入 & 下一个时钟周期是执行阶段的第几个时钟周期 \\
		\hline
		hilo$\_$o & 64 & 输出 & 保存的乘法结果 \\
		\hline
		cnt$\_$o & 2 & 输出 & 当前处于执行阶段的第几个时钟周期 \\
		\hline
	\end{tabular}
\end{table}
\subsection{访存阶段}
流水线的访存阶段包括MEM、MEM/WB两个模块。
\subsubsection{MEM模块}
\paragraph{功能描述}
\quad

\quad

如果是加载、存储指令,那么会对数据存储器进行访问。此外,还会在该模块进行异常判断。对应mem.v文件。其实现为组合逻辑电路。

\begin{enumerate}
	\item 加载指令实现思路
	
	加载指令在译码阶段进行译码,得到运算类型alusel$\_$o、aluop$\_$o,以及要写的目的寄存器信息。这些信息传递到执行阶段,然后又传递到访存阶段,访存阶段依据这些信息,设置对数据存储器RAM的访问信号。从RAM读取回来的数据需要按照加载指令的类型、加载地址进行对齐调整,调整后的结果作为最终要写入目的寄存器的数据。
	
	\item 存储指令实现思路
	
	存储指令在译码阶段进行译码,得到运算类型alusel$\_$o、aluop$\_$o,以及要存储的数据。这些信息传递到执行阶段,然后又传递到访存阶段,访存阶段依据这些信息,设置对数据存储器RAM的访问信号,将数据写入RAM。
\end{enumerate}

\paragraph{端口说明}
\quad

\quad
\begin{table}[H]
	\centering
	\caption{MEM模块的接口描述}
	\begin{tabular}{|l|l|l|l|}
		\hline
		接口名 & 宽度(bit) & 输入/输出 & 作用 \\
		\hline
		rst & 1 & 输入 & 复位信号 \\
		\hline
		wd$\_$i & 5 & 输入 & 访存阶段的指令要写入的目的寄存器地址 \\
		\hline
		wreg$\_$i & 1 & 输入 & 访存阶段的指令是否有要写入的目的寄存器 \\
		\hline
		wdata$\_$i & 32 & 输入 & 访存阶段的指令要写入目的寄存器的值 \\
		\hline
		wd$\_$o & 5 & 输出 & 访存阶段的指令最终要写入的目的寄存器地址 \\
		\hline
		wreg$\_$o & 1 & 输出 & 访存阶段的指令最终是否有要写入的目的寄存器 \\
		\hline
		wdata$\_$o & 32 & 输出 & 访存阶段的指令最终要写入目的寄存器的值 \\
		\hline
		aluop$\_$i & 8 & 输入 & 访存阶段的指令要进行的运算的子类型 \\
		\hline
		mem$\_$addr$\_$i & 32 & 输入 & 访存阶段的加载、存储指令对应的存储器地址 \\
		\hline
		\multirow{2}{*}{reg2$\_$i} & \multirow{2}{*}{32} & \multirow{2}{*}{输入} & 访存阶段的存储指令要存储的数据,或者lwl、lwr \\
		& & & 指令要写入的目的寄存器的原始值 \\
		\hline
		mem$\_$data$\_$i & 32 & 输入 & 从数据存储器读取的数据 \\
		\hline
		mem$\_$addr$\_$o & 32 & 输出 & 要访问的数据存储器的地址 \\
		\hline
		mem$\_$we$\_$o & 1 & 输出 & 是否是写操作,为1表示是写操作 \\
		\hline
		mem$\_$sel$\_$o & 4 & 输出 & 字节选择信号 \\
		\hline
		mem$\_$data$\_$o & 32 & 输出 & 要写入数据存储器的数据 \\
		\hline
		mem$\_$ce$\_$o & 1 & 输出 & 数据存储器使能信号 \\
		\hline
		whilo$\_$i & 1 & 输入 & 访存阶段的指令是否要写HI、LO寄存器 \\
		\hline
		hi$\_$i & 32 & 输入 & 访存阶段的指令要写入HI寄存器的值 \\
		\hline
		lo$\_$i & 32 & 输入 & 访存阶段的指令要写入LO寄存器的值 \\
		\hline
		whilo$\_$o & 1 & 输出 & 访存阶段的指令最终是否要写HI、LO寄存器 \\
		\hline
		hi$\_$o & 32 & 输出 & 访存阶段的指令最终要写入HI寄存器的值 \\
		\hline
		lo$\_$o & 32 & 输出 & 访存阶段的指令最终要写入LO寄存器的值 \\
		\hline
		cp0$\_$reg$\_$we$\_$i & 1 & 输入 & 访存阶段的指令是否要写CP0中的寄存器 \\
		\hline
		cp0$\_$reg$\_$write$\_$addr$\_$i & 5 & 输入 & 访存阶段的指令要写的CP0中寄存器的地址 \\
		\hline
		cp0$\_$reg$\_$data$\_$i & 32 & 输入 & 访存阶段的指令要写入CP0中寄存器的数据 \\
		\hline
		cp0$\_$reg$\_$we$\_$o & 1 & 输出 & 访存阶段的指令最终是否要写CP0中的寄存器 \\
		\hline
		cp0$\_$reg$\_$write$\_$addr$\_$o & 5 & 输出 & 访存阶段的指令最终要写的CP0中寄存器的地址 \\
		\hline
		cp0$\_$reg$\_$data$\_$o & 32 & 输出 & 访存阶段的指令最终要写入CP0中寄存器的数据 \\
		\hline
	\end{tabular}
\end{table}
\begin{table}[H]
	\centering
	\begin{tabular}{|l|l|l|l|}
		\hline
		excepttype$\_$i & 32 & 输入 & 译码、执行阶段收集到的异常信息 \\
		\hline
		current$\_$inst$\_$address & 32 & 输入 & 访存阶段指令的地址 \\
		\hline
		is$\_$in$\_$delayslot$\_$i & 1 & 输入 & 访存阶段的指令是否是延迟槽指令 \\
		\hline
		cp0$\_$status$\_$i & 32 & 输入 & CP0中Status寄存器的值 \\
		\hline
		cp0$\_$cause$\_$i & 32 & 输入 & CP0中Cause寄存器的值 \\
		\hline
		cp0$\_$epc$\_$i & 32 & 输入 & CP0中EPC寄存器的值 \\
		\hline
		wb$\_$cp0$\_$reg$\_$we & 1 & 输入 & 回写阶段的指令是否要写CP0中的寄存器 \\
		\hline
		wb$\_$cp0$\_$reg$\_$write$\_$address & 5 & 输入 & 回写阶段的指令要写的CP0中寄存器的地址 \\
		\hline
		wb$\_$cp0$\_$reg$\_$data & 32 & 输入 & 回写阶段的指令要写入CP0中寄存器的值 \\
		\hline
		excepttype$\_$o & 32 & 输出 & 最终的异常类型 \\
		\hline
		current$\_$inst$\_$address$\_$o & 32 & 输出 & 访存阶段指令的地址 \\
		\hline
		is$\_$in$\_$delayslot$\_$o & 1 & 输出 & 访存阶段的指令是否是延迟槽指令 \\
		\hline
		cp0$\_$epc$\_$o & 32 & 输出 & CP0中EPC寄存器的最新值 \\
		\hline
	\end{tabular}
\end{table}
\subsubsection{MEM/WB模块}
\paragraph{功能描述}
\quad

\quad

实现访存与回写阶段之间的寄存器,将访存阶段的结果暂时保存,在下一个时钟周期的上升沿传递到回写阶段,对应mem$\_$wb.v文件。
\paragraph{端口说明}
\quad

\quad
\begin{table}[H]
	\centering
	\caption{MEM/WB模块的接口描述}
	\begin{tabular}{|l|l|l|l|}
		\hline
		接口名 & 宽度(bit) & 输入/输出 & 作用 \\
		\hline
		rst & 1 & 输入 & 复位信号 \\
		\hline
		clk & 1 & 输入 & 时钟信号 \\
		\hline
		mem$\_$wd & 5 & 输入 & 访存阶段的指令最终要写入的目的寄存器地址 \\
		\hline
		mem$\_$wreg & 1 & 输入 & 访存阶段的指令最终是否有要写入的目的寄存器 \\
		\hline
		mem$\_$wdata & 32 & 输入 & 访存阶段的指令最终要写入目的寄存器的值 \\
		\hline
		wb$\_$wd & 5 & 输出 & 回写阶段的指令要写入的目的寄存器地址 \\
		\hline
		wb$\_$wreg & 1 & 输出 & 回写阶段的指令是否有要写入的目的寄存器 \\
		\hline
		wb$\_$wdata & 32 & 输出 & 回写阶段的指令要写入目的寄存器的值 \\
		\hline
		mem$\_$cp0$\_$reg$\_$we & 1 & 输入 & 访存阶段的指令是否要写CP0中的寄存器 \\
		\hline
		mem$\_$cp0$\_$reg$\_$write$\_$addr & 5 & 输入 & 访存阶段的指令要写的CP0中寄存器的地址 \\
		\hline
		mem$\_$cp0$\_$reg$\_$data & 32 & 输入 & 访存阶段的指令要写入CP0中寄存器的数据 \\
		\hline
		wb$\_$cp0$\_$reg$\_$we & 1 & 输出 & 回写阶段的指令是否要写CP0中的寄存器 \\
		\hline
		wb$\_$cp0$\_$reg$\_$write$\_$addr & 5 & 输出 & 回写阶段的指令要写的CP0中寄存器的地址 \\
		\hline
		wb$\_$cp0$\_$reg$\_$data & 32 & 输出 & 回写阶段的指令要写入CP0中寄存器的数据 \\
		\hline
		mem$\_$whilo & 1 & 输入 & 访存阶段的指令是否要写HI、LO寄存器 \\
		\hline
		mem$\_$hi & 32 & 输入 & 访存阶段的指令要写入HI寄存器的值 \\
		\hline
		mem$\_$lo & 32 & 输入 & 访存阶段的指令要写入LO寄存器的值 \\
		\hline
		wb$\_$whilo & 1 & 输出 & 回写阶段的指令是否要写HI、LO寄存器 \\
		\hline
		wb$\_$hi & 32 & 输出 & 回写阶段的指令要写入HI寄存器的值 \\
		\hline
		wb$\_$lo & 32 & 输出 & 回写阶段的指令要写入LO寄存器的值 \\
		\hline
		stall & 1 & 输入 & 访存阶段是否暂停 \\
		\hline
		flush & 1 & 输入 & 是否清除流水线 \\
		\hline
	\end{tabular}
\end{table}
\subsection{回写阶段}
\subsubsection{CP0模块}
\paragraph{功能描述}
\quad

\quad

对应MIPS架构中的协处理器CP0。
\paragraph{端口说明}
\quad

\quad
\begin{table}[H]
	\centering
	\caption{CP0模块的接口描述}
	\begin{tabular}{|l|l|l|l|}
		\hline
		接口名 & 宽度(bit) & 输入/输出 & 作用 \\
		\hline
		rst & 1 & 输入 & 复位信号 \\
		\hline
		clk & 1 & 输入 & 时钟信号 \\
		\hline
		raddr$\_$i & 5 & 输入 & 要读取的CP0中寄存器的地址 \\
		\hline
		int$\_$i & 6 & 输入 & 6个外部硬件中断输入 \\
		\hline
		we$\_$i & 1 & 输入 & 是否要写CP0中的寄存器 \\
		\hline
		waddr$\_$i & 5 & 输入 & 要写的CP0中寄存器的地址 \\
		\hline
		wdata$\_$i & 32 & 输入 & 要写入CP0中寄存器的数据 \\
		\hline
		data$\_$o & 32 & 输出 & 读出的CP0中某个寄存器的值 \\
		\hline
		count$\_$o & 32 & 输出 & Count寄存器的值 \\
		\hline
		compare$\_$o & 32 & 输出 & Compare寄存器的值 \\
		\hline
		status$\_$o & 32 & 输出 & Status寄存器的值 \\
		\hline
		cause$\_$o & 32 & 输出 & Cause寄存器的值 \\
		\hline
		epc$\_$o & 32 & 输出 & EPC寄存器的值 \\
		\hline
		config$\_$o & 32 & 输出 & Config寄存器的值 \\
		\hline
		prid$\_$o & 32 & 输出 & PRId寄存器的值 \\
		\hline
		timer$\_$int$\_$o & 1 & 输出 & 是否有定时中断发生 \\
		\hline
		excepttype$\_$i & 32 & 输入 & 最终的异常类型 \\
		\hline
		current$\_$inst$\_$address$\_$i & 32 & 输入 & 发生异常的指令地址 \\
		\hline
		is$\_$in$\_$delayslot$\_$i & 1 & 输入 & 发生异常的指令是否是延迟槽指令 \\
		\hline
	\end{tabular}
\end{table}
\subsubsection{HILO模块}
\paragraph{功能描述}
\quad

\quad

实现寄存器HI、LO,在乘法指令的处理过程中会使用到这两个寄存器。
\paragraph{端口说明}
\quad

\quad
\begin{table}[H]
	\centering
	\caption{HILO模块的接口描述}
	\begin{tabular}{|l|l|l|l|}
		\hline
		接口名 & 宽度(bit) & 输入/输出 & 作用 \\
		\hline
		rst & 1 & 输入 & 复位信号 \\
		\hline
		clk & 1 & 输入 & 时钟信号 \\
		\hline
		we & 1 & 输入 & HI、LO寄存器写使能信号 \\
		\hline
		hi$\_$i & 32 & 输入 & 要写入HI寄存器的值 \\
		\hline
		lo$\_$i & 32 & 输入 & 要写入LO寄存器的值 \\
		\hline
		hi$\_$o & 32 & 输出 & HI寄存器的值 \\
		\hline
		lo$\_$o & 32 & 输出 & LO寄存器的值 \\
		\hline
	\end{tabular}
\end{table}
\subsection{控制部件}
\subsubsection{CTRL模块}
\paragraph{功能描述}
\quad

\quad

控制整个流水线的暂停、清除等动作,所以不便于将其归入流水线的某一个阶段,对应ctrl.v文件。

\begin{enumerate}[(1)]
\item 流水线暂停机制的设计与实现:

	OpenMIPS在进行异常处理和load相关发生冲突时,需要暂停流水线,一种直观的实现方法是:要暂停流水线,只需保持取指令地址PC的值不变,同时保持流水线各个阶段的寄存器(也就是IF/ID、ID/EX、EX/MEM、MEM/WB模块的输出)不变。

	OpenMIPS采用的是一种改进的方法:假如位于流水线第n阶段的指令请求流水线暂停,那么需保持取指令地址PC的值不变,同时保持流水线第n阶段、第n阶段之前的各个阶段的寄存器不变,而第n阶段后面的指令继续运行。比如:流水线执行阶段的指令请求流水线暂停,那么保持PC不变,同时保持取指、译码、执行阶段的寄存器不变,但是可以允许访存、回写阶段的指令继续运行。

	为此,设计添加CTRL模块,其作用是接收各个阶段传递过来的流水线暂停请求信号,从而控制流水线各个阶段的运行。具体设计如下:

	CTRL模块的输入来自ID、EX模块的请求暂停信号stallreq,对于OpenMIPS而言,只有译码、执行阶段可能会有暂停请求,取指、访存阶段都没有暂停请求,因为指令读取、数据存储器的读写操作都可以在一个时钟周期完成。

	CTRL模块对暂停请求信号进行判断,然后输出流水线暂停信号stall。stall输出到PC、IF/ID、ID/EX、 EX/MEM、MEM/WB等模块,从而控制PC的值,以及流水线各个阶段的寄存器。

	其具体实现方法如下:
	\begin{enumerate}[(a)]
		\item 当处于流水线执行阶段的指令请求暂停时(即stallreq$\_$from$\_$ex等于Stop),按照OpenMIPS流水线暂停机制的设计,要求取指、译码、执行阶段暂停,而访存、回写阶段继续,所以设置stall为6'b001111。
	
		\item 当处于流水线译码阶段的指令请求暂停时(即stallreq$\_$from$\_$id等于Stop),按照OpenMIPS流水线暂停机制的设计,要求取指、译码阶段暂停,而执行、访存、回写阶段继续,所以设置stall为6'b000111。
	
		\item 其余情况下,设置stall为6'b000000,表示不暂停流水线。
	\end{enumerate}
\end{enumerate}
\paragraph{端口说明}
\quad

\quad
\begin{table}[H]
	\centering
	\caption{CTRL模块的接口描述}
	\begin{tabular}{|l|l|l|l|}
		\hline
		接口名 & 宽度(bit) & 输入/输出 & 作用 \\
		\hline
		rst & 1 & 输入 & 复位信号 \\
		\hline
		stallreq$\_$from$\_$id & 1 & 输入 & 处于译码阶段的指令是否请求流水线暂停信号 \\
		\hline
		stallreq$\_$from$\_$ex & 1 & 输入 & 处于执行阶段的指令是否请求流水线暂停信号 \\
		\hline
		stall & 6 & 输出 & 暂停流水线控制信号 \\
		\hline
		cp0$\_$epc$\_$i & 32 & 输入 & EPC寄存器的最新值 \\
		\hline
		excepttype$\_$i & 32 & 输入 & 最终的异常类型 \\
		\hline
		new$\_$pc & 32 & 输出 & 异常处理入口地址 \\
		\hline
		flush & 1 & 输出 & 是否清除流水线 \\
		\hline
	\end{tabular}
\end{table}
